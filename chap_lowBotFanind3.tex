\chapter{Depth three circuits of low bottom fan-in}

Kayal and Saha~\cite{KayalSaha14} show that the technique of projected shifted partial derivatives can also be used to prove lower bounds for subclasses of non-homogeneous depth three circuits, namely depth three circuits with \emph{bounded bottom fan-in}.
We shall denote the class of depth three circuits of bottom fan-in bounded by $r$ as $\SPS^{[r]}$ circuits.
The ideas involved have also been useful in addressing depth-$4$ circuits with ``low-arity'' \cite{KumarSaraf15,KayalSaha15} that we shall see later. 

\section{$\SPS$ circuits with bottom fan-in $O(\sqrt{d})$}

Now let us focus on $\SPS^{[r]}$ circuits, where all linear polynomials in the circuit depend on at most $r$ variables.
The following is the key observation of Kayal and Saha \cite{KayalSaha14} and can be verified easily from the proof of \autoref{lem:d3-d5}.

\begin{observation}[\cite{KayalSaha14}]\label{obs:d3-d5-fanin}
  Starting with a $\SPS^{[r]}$ circuit $C$ of size $s$ computing a homogeneous $n$-variate polynomial of degree $d$, the resulting $\Sigma\Pi\Sigma\mywedge\Sigma$ circuit $C'$ obtained from \autoref{lem:d3-d5} is in fact a $\Sigma\Pi\Sigma\mywedge\Sigma^{[r]}$ circuit of size $s' = \poly(s) \cdot 2^{O(\sqrt{d})}$.

  Thus, by expanding the all powers of linear polynomials computed in the bottom two layers of the $\Sigma\Pi\Sigma\mywedge\Sigma^{[r]}$ circuit $C'$, the circuit $C'$ can be rewritten as a homogeneous depth $4$ circuit of bottom support bounded by $r$ and size $s'' \leq s' \cdot d^r$
\end{observation}

This observation in combination with \autoref{thm:KLSS-lowsupp} immediately yields the main theorem of \cite{KayalSaha14}.

\begin{theorem}[\cite{KayalSaha14}]\label{thm:kaysaha-main}
  Over any characteristic zero field $\F$, any $\SPS^{[r]}$ circuit
  $C$ computing the polynomial $\IMM_{n,d}$, for suitably chosen
  parameters $n$ and $d$ with $n = d^{O(1)}$, must have size $s =
  n^{\Omega(d/r)}$. 
\end{theorem}


\section{$\SPS$ circuits with bottom fan-in $n^{1-\epsilon}$}

Kayal and Saha \cite{KayalSaha14} also prove lower bounds for depth three circuits
where the bottom fan-in is bounded away from $n$ by any polynomial factor. 

\begin{theorem}[\cite{KayalSaha14}] Let $\epsilon > 0$ be any
  constant.
Then over any characteristic zero field, there exists an explicit polynomial $P$ such that any $\SPS^{[n^{1 - \epsilon}]}$ circuit computing $P$ must have size $n^{\Omega_\epsilon(\sqrt{d})}$.
\end{theorem}

We shall work with $\epsilon = 0.1$ to save on some variables. All the ideas here can be made to work for any $\epsilon > 0$. 

As a first step, we shall start with a $\SPS^{[n^{0.9}]}$ circuit and use \autoref{obs:d3-d5-fanin} to convert it to a homogeneous $\Sigma\Pi\Sigma\mywedge\Sigma^{[n^{0.9}]}$ circuit computing the same polynomial.
As we have just seen, if the fan-in of all the linear polynomials at the bottom were instead $O(\sqrt{d})$, then we can directly apply \autoref{obs:d3-d5-fanin} and use \autoref{thm:KLSS-lowsupp} to prove the lower bound via projected shifted partial derivatives.
What we would like to do is reduce to this case somehow, and a natural approach is to use random restrictions.

\subsection*{Attempt 1}


Assume that the size $s$ of the circuit is at most $n^{0.1\sqrt{d}}$.
Thus there are at most $n^{0.1 \sqrt{d}}$ linear polynomials computed by the homogeneous depth-$5$ circuit as the bottom layer, each of which is a linear polynomial on just $n^{0.9}$ variables.
Let us apply a random restriction of setting each variable independently to zero with probability $1 - \alpha = 1 - \frac{\sqrt{d}}{n^{0.9}}$, so that the expected size of a linear form is now $\sqrt{d}$.

What would Chernoff's bound give us?
We have $n^{0.9}$ independent variables that takes value $1$ with probability $\alpha$ indicating whether that variable is kept alive or not.
Therefore the expected number of variables kept alive in that linear form is $\mu = \sqrt{d}$.
Chernoff's bound states that the probability that more than $\mu (1 + \delta)$ of them are kept alive is at most $\exp(-\delta^2 \mu) = \exp(-O(\sqrt{d}))$ if $\delta$ is a constant.
This is not enough for a union bound over $s = n^{0.1\sqrt{d}}$ gates.

Nevertheless, let us do what we can and use a random restriction that sets variables to zero with probability $1 - \alpha = 1 - \frac{d}{n^{0.9}}$.
Now we can indeed use Chernoff's bound to show that with very high probability, all linear polynomials are now over at most $2d$ variables as the expected size of the linear form is $d$.

The number of total variables that are alive $n'$ is still about $n^{0.1}$ which is hopefully enough.
We shall get to this point later, and for now hope that we can make $n' \gg d$. 

Thus we now need to deal with a $\Sigma\Pi\Sigma\mywedge\Sigma^{d}$ homogeneous circuit (dropping the constant factor for brevity).
The question now is, have we made any progress at all?


\subsection*{Attempt 2}

Since the top gate is a $+$ gate, it suffices to understand one term 
\[
T \spaced{=} \prod_{i} \inparen{\sum_j \ell_{ij}^{d_i}}
\]
Some of the $d_i$s could be large specifically $d_i > 2\sqrt{d}$ and others smaller. We shall split the above product as
\begin{eqnarray*}
T &=& \prod_{\substack{i\\d_i \leq 2\sqrt{d}}} \inparen{\sum_j \ell_{ij}^{d_i}} \; \cdot \; \prod_{\substack{i\\d_i > 2\sqrt{d}}} \inparen{\sum_j \ell_{ij}^{d_i}}\\
  & = & T_{\text{low}} \; \cdot \; T_{\text{high}}
\end{eqnarray*}
We already have some experience with handling low degree factors so we only need to address the high degree factors.
It would be great if we could just expand out the high-degree factors and show that they do not contribute anything but at first sight it seems like there were would too many monomials for this to work.
But Kayal and Saha \cite{KayalSaha14} show a way to handle this delicately. 

To get a sense of this, let us just take one power of a linear form $\ell^{d_i}$. All we know right now is that $\ell$ depends on $d$ variables, and $2\sqrt{d} < d_i \leq d$. Expanding this na\"ively would yield about $d^d$ monomials but let us count this a little differently. Let us write this as $\ell^{2\sqrt{d}} \cdot \ell^{d_i - 2\sqrt{d}}$ and expand only the $\ell^{2\sqrt{d}}$ part. This gives an expression of the form
\[
\ell^{d_i} = \sum_{i=1}^t m_i \cdot \ell^{d_i - 2\sqrt{d}}
\]
with $t \leq d^{2\sqrt{d}}$. Some of the $m_i$s could be non-multiquadratic so they would not contribute anything to the projected shifted partial derivatives. But the $m_i$s that are multiquadratic must therefore have support size at least $\sqrt{d}$. Since these are all monomials over just $d$ variables, the number of distinct multiquadratic monomials is at most $d^{\sqrt{d}}$. If we can eliminate all these monomials, we would be done. Let $\mathcal{B}$ be the set of all multiquadratic monomials of $\ell_i^{2\sqrt{d}}$ when we run over all the $\ell_i$s in the circuit. After the random restriction from the previous step, we have $n' \approx n^{0.1}$ variables alive. Therefore, we have that $\abs{\mathcal{B}} \leq (n')^{0.1\sqrt{d}} \cdot d^{\sqrt{d}} \leq (n')^{0.51 \sqrt{d}}$ since we will definitely have $n' \geq d^2$. 

To eliminate all monomials in $\mathcal{B}$, we use a random restriction $\rho_\beta$ and set each variable to zero with probability $1 - \beta$ where $\beta = \frac{1}{(n')^{0.6}}$. For such a restriction, the probability that a fixed $\sqrt{d}$ support monomial survives is at most $(n')^{-0.6\sqrt{d}}$ and thus even after a union bound, with overwhelming probability all monomials in $\mathcal{B}$ are now set to zero. What this means is that the only monomials in $T_{\text{high}}$ are non-multiquadratic monomials and hence will not contribute to the projected shifted partials. 
\[
\rho_\beta(\SPS\mywedge\Sigma^{d}) \spaced{=} \SPSP^{[2\sqrt{d}]} \;+\; (\text{non-multiquadratic monomials})
\]

Combining it all together, if $f$ is a polynomial computed by a $\SPS^{[n^{0.9}]}$ circuit, then $\rho_\beta \circ \rho_\alpha (f)$ satisfies
\[
\CM{PSD}_{k,\ell}\inparen{\rho_\beta \circ \rho_\alpha(f)} \spaced{\leq} \binom{2\sqrt{d}}{k} \cdot \binom{n'}{\ell + 2k\sqrt{d}}
\]
where $n'$ is the variables alive in $\rho_\beta \circ \rho_\alpha(f)$. 

\subsection*{Constructing the hard polynomial}

What do the two restrictions together look like? The first restriction keeps a variable alive with probability $\alpha = \frac{d}{n^{0.9}}$ which leaves about $n' = \frac{n^{0.1}}{d}$ variables alive. The second restrictions keeps a variable alive with probability $\beta = (n')^{-0.6}$. Hence together, this can be viewed as a single random restriction that keeps a variable alive with probability about
\[
\alpha \cdot \beta \spaced{=} \frac{d}{n^{0.9}} \cdot \frac{1}{\inparen{\frac{n^{0.1}}{d}}^{0.6}} \spaced{=} \frac{d^{1.6}}{n^{0.96}}.
\]
We'll build our hard polynomial so that $n \geq d^{1000}$ so we can think of the two random restrictions as a single restriction $\rho_\epsilon$ where $\epsilon = \frac{1}{n^{0.99}}$.\\

The question now is, can we build a hard polynomial $P$ on $n$ variables with degree $d$ satisfying $n > d^{1000}$ such that \emph{even after} a random restriction $\rho_\epsilon$ for $\epsilon = n^{-0.99}$, the polynomial $\rho_\epsilon(P)$ has a large dimension of projected shifted partial derivatives? The answer is indeed `Yes' and the $\NW$ polynomial family can be made robust using the linear blow-up trick (\autoref{lem:lin-transform-trick}). 

Fix some parameter $d$ and choose $m,e$ as in \autoref{lem:d4hom-goldilocks-LB} so that $\NW_{d,m,e}$ has nearly maximal dimension of projected shifted partial derivatives. Now consider the hard polynomial to be $\NW_{d,m,e} \circ \mathrm{Lin}$ obtained by replacing each variable of $\NW_{d,m,e}$ by a sum of $d^{1000}$ fresh variables. Hence we are now in a setting where we have an $n$ variate polynomial of degree $d$ with $n \geq d^{1000}$. \\

Now suppose this polynomial is computed by a $\SPS^{[n^{0.9}]}$ circuit. Applying $\rho_\epsilon$, we get that $\rho_\epsilon(\NW_{d,m,e} \circ \mathrm{Lin})$ is computed by a circuit of the form 
\[
\SPSP^{[2\sqrt{d}]} + (\text{non-multiquadratic}).
\]
Since $\rho_\epsilon(\NW_{d,m,e})$ has a copy of $\NW_{d,m,e}$ sitting inside, by setting additional variables to zero, we now have a circuit of the same form as above computing $\NW_{d,m,e}$ and we have our lower bound from \autoref{lem:upper-bound-low-supp} and \autoref{lem:d4hom-goldilocks-LB}:
\[
\CM{PSD}_{k,\ell}\inparen{\SPSP^{[2\sqrt{d}]} + \text{non-multiquadratic}} =  \CM{PSD}_{k,\ell}\inparen{\SPSP^{[2\sqrt{d}]}} \ll \CM{PSD}_{k,\ell}(\NW_{d,m,e}). 
\]

This completes the proof of \autoref{thm:kaysaha-main} (it should be clear that we can run the whole argument with $\IMM$ instead of $\NW$). \qed\\

\begin{exercise}
Use the techniques to prove a similar $n^{\Omega(\sqrt{d})}$ lower bound for homogeneous $\SPSP\Sigma^{[n^{0.99}]}$ circuits. Find out where the above proof breaks and see if there are different sets of parameters to fix the issue. 
\end{exercise}









%%% Local Variables:
%%% mode: latex
%%% TeX-master: "fancymain"
%%% End:
