\chapter{Separation between monotone $\VP$ and $\VNP$}
\label{chap:monVPVNP}

\newcommand{\monVP}{\mathrm{mon}\text{-}\VP}
\newcommand{\monVNP}{\mathrm{mon}\text{-}\VNP}
\newcommand{\Row}{\operatorname{Row}}
\newcommand{\Col}{\operatorname{Col}}


We have seen that monotone computational models are rather restrictive and we have seen lower bounds that use just sparsity as the complexity measure (with a clever monomial counting making use of the restrictive nature of computatation). We have seen exponential lower bounds against monotone circuits (\autoref{thm:monotone-circuit-lbs}) and also a quasipolynomial separation between monotone ABPs and monotone circuits (\autoref{thm:mon-abp-ckt-sep}). Here, we shall see a beautiful result of Yehudayoff~\cite{Yeh18} that separates $\VP$ and $\VNP$ in the monotone world.

The definition of the class ``monotone $\VP$'' is clear. The following is an analogous definition of the class monotone $\VNP$ (or $\monVP$). Throughout this chapter, we will restrict ourselves to polynomials over $\R$. Over a characteristic zero field such as $\R$, we will work with the \emph{standard} definition of monotone models which is just that there are no negative constants. 

\begin{definition}[Monotone $\VNP$] A family of polynomial $\set{f_n(\vecx)} \in \R[\vecx]$ is said to be in monotone $\VNP$ (or $\monVNP$) if there is a family of polynomials $\set{g_n(\vecx, \vecy)} \in \monVP$ with $\abs{\vecy} = \poly{\abs{\vecx}}$ such that
  \[
    f(\vecx) = \sum_{\vecb \in \set{0,1}^{\abs{\vecy}}} g(\vecx, \vecb).\qedhere
  \]
\end{definition}

We are now ready to state the main theorem.

\begin{theorem}[\cite{Yeh18}]\label{thm:monVPVNPseparation}
There is an explicit polynomial family $\set{f_n} \in \monVNP$ such that any $\monVP$ circuit computing it must have size $2^{\Omega(n/\log n)}$. 
\end{theorem}

\subsection*{The importance of coefficients}

We already know lower bounds against monotone circuits. Can we not find a small modification such that the lower bound is using a polynomial that is in $\monVNP$? The issue with this is that all the techniques for monotone lower bounds we have seen so far \emph{only uses the structure of the set of monomials} and \emph{not} the coefficients. Any technique that uses just the underlying set of monomials to prove a lower bound against $\monVP$ also necessarily yields a lower bound against $\monVNP$.

\begin{exercise}
  Let $\Gamma:\F[\vecx] \rightarrow \R$ be a complexity measure such that
  \begin{itemize}
  \item For two polynomials $f,g$ with the same set of monomials, $\Gamma(f) = \Gamma(g)$,
  \item For any $f \in \VP$, we have $\Gamma(f)$ is ``small''.
  \end{itemize}
  Show that $\Gamma(h)$ is small for every $h \in \monVNP$ as well. 
\end{exercise}

Therefore, it is imperative that any separation between $\monVP$ and $\monVNP$ must necessarily use the coefficients of the underlying polynomials.

\subsection*{The polynomial}

\begin{align*}
  P(\vecx) & := \frac{1}{2^n} \sum_{\vecb \in \set{0,1}^n} \prod_{i=1}^n \sum_{j=1}^n b_j x_{ij}\\
           & = \sum_{\sigma: [n] \rightarrow [n]} \frac{1}{2^{\abs{\Range(\sigma)}}} \cdot x_{1\sigma(1)} \cdots x_{n\sigma(n)}
\end{align*}

\begin{lemma}
  The polynomial $P \in \monVNP$. 
\end{lemma}
\begin{proof}
  Follows just from the definition. 
\end{proof}

Before we proceed towards describing the complexity measure, a few notations would help the writing. 

\subsection*{Some notation}

The polynomial $P(\vecx)$ is set-multilinear respect to rows, and any monotone model computing this must also be set-multilinear. Therefore, all intermediate computations must involve monomials that picks at most $1$ variable from each row.

For a set-multilinear monomial $\alpha$, let $\Row(\alpha)$ denote the set of rows that it picks up variables from and similarly define $\Col(\alpha)$ analogously.

We shall say a polynomial $f(\vecx)$ is \emph{row-aligned} if $\Row(\alpha)$ is the same for every $\alpha \in f$. In such a case, define $\Row(f) = \Row(\alpha)$ for an arbitrary $\alpha \in f$. \\


\section{Structural weakness of monotone circuits}

We have already seen the standard structural weakness for monotone circuits that was used in both \autoref{sec:monotone-ckts} and \autoref{chap:monABPCktSep}. We will use this specifically for a circuit computing a set-multilinear polynomial such as $P$. 

\begin{lemma}\label{lem:monVPVNP:structure}
  Let $C$ be a monotone circuit of size $s$ computing the polynomial $P(\vecx)$ defined above. Then, $P(\vecx)$ can be written as
  \begin{equation}\label{eqn:monVPVNP-main}
    P(\vecx) = \sum_{i=1}^s f_i(\vecx) \cdot g_i(\vecx)
  \end{equation}
  such that
  \begin{itemize}\itemsep 0pt
  \item For each $i$, we have that $f_i$ and $g_i$ are row-aligned with
    $\Row(f_i) \sqcup \Row(g_i) = [n]$. Also, $\frac{n}{3} \leq \Row(f_i), \Row(g_i) \leq \frac{2n}{3}$. 
  \item $f_i$'s and $g_i$'s contain only non-negative coefficients,
  \item for any $i$, and any $\alpha \in f_i$ and $\beta\in g_i$, we have
    \[
      f_i[\alpha]\cdot g_i[\beta] \leq P[\alpha \cdot \beta]
    \]
    where by $h[\alpha]$ we denote the coefficient of the monomial $\alpha$ in $h$.
  \end{itemize}
\end{lemma}

\noindent
The proof of this is exactly along the same lines as we have seen earlier and we skip it. We can now describe the complexity measure.

We shall choose a parameter $\delta  = O\inparen{\frac{1}{\log n}}$. Let $\Pi:\R[\vecx] \rightarrow \R[\vecx]$ be the map that projects down to only monomials that touch exactly $\delta n$ columns. That is,
\[
  \Pi(h) = \sum_{\substack{\alpha \in h \\ \abs{\Col(\alpha)} = \delta n}} h[\alpha] \cdot \alpha
\]

\paragraph{The complexity measure:} Define $\Gamma:\R[\vecx] \rightarrow \R$ as
\begin{align*}
  \Gamma(h(\vecx)) & := \abs{\Pi(h)}_1 \\
                   & = \sum_{\substack{\alpha \in h \\ \abs{\Col(\alpha)} = \delta n}} h[\alpha],
\end{align*}
the sum of the coefficients of monomials that touch exactly $\delta n$ columns. \\

Given the above complexity measure, it would be useful to also have one more notation. Let $\mathcal{F}_{n,k}$ denote the set of onto functions from $[n]$ to $[k]$, and let $F_{n,k}$ be the size of this set. We would be mostly interested in $k = \delta n$.

\begin{proposition}\label{prop:bound-on-onto-fns}
  For $\delta = O\inparen{\frac{1}{\log n}}$, we have that
  \[
    \frac{1}{2}\cdot (\delta n)^n \leq  F_{n,\delta n} \leq (\delta n)^n
  \]
\end{proposition}
\begin{proof-sketch}
  The upper bound is clear. As for the lower bound, notice that for $\delta$ so small, a random function from $[n]$ to $[\delta n]$ would be onto with probability at least $1/2$. 
\end{proof-sketch}

\begin{lemmawp}(Upper bound for the polynomial) \label{lemma:monVPVNP-ub-for-poly} For the polynomial defined above, 
  \[
    \Gamma(P) = \binom{n}{\delta n} \cdot  2^{-\delta n} \cdot F_{n,k}. 
  \]
\end{lemmawp}

\section{Upper bounding measure on building blocks}

The strategy is the usual plan of bounding the measure for each building block in \eqref{eqn:monVPVNP-main}. The following is the main technical lemma.

\begin{lemma}[Main technical lemma] \label{lem:monVPVNP-weakness-lemma}
  Let $f(\vecx)$ and $g(\vecx)$ be row-aligned polynomials such that
  \begin{itemize}\itemsep 0pt
  \item $\Row(f) \sqcup \Row(g) = [n]$,
  \item $\frac{n}{3} \leq \Row(f),\Row(g) \leq \frac{2n}{3}$,
  \item For each $\alpha \in f$ and $\beta \in g$, we have $f[\alpha] \cdot g[\beta] \leq P(\alpha \cdot \beta)$.
  \end{itemize}
  Then,
  \[
    \Gamma(f \cdot g) \leq 2^{-\Omega(\delta n)} \cdot \Gamma(P).
  \]
\end{lemma}

Clearly \autoref{thm:monVPVNPseparation} follows readily from the above lemma.
The rest of this chapter will be spent on the proof of this lemma. To make it a little easier to follow, we denote certain terms in \textcolor{BrickRed}{red} to indicate that the primary gain is coming from that term.
Let $\epsilon = 1/20$. We will call a monomial $\alpha$ to be ``thin'' if $\abs{\Col(\alpha)} \leq (1-\epsilon) \delta n$, and ``wide'' otherwise. 
We shall split $\Gamma(f \cdot g)$ into three sums.
\begin{align*}
  \Gamma(f \cdot g) & :=   \sum_{S \in \binom{[n]}{\delta n}} \sum_{\substack{\alpha, \beta\\ \Col(\alpha \cdot \beta) = S}} f[\alpha] \cdot g[\beta]\\
                    & \leq \sum_{S \in \binom{[n]}{\delta n}} \textcolor{BrickRed}{\sum_{\substack{\alpha, \beta\\ \Col(\alpha \cdot \beta) = S \\ \text{$\alpha$ thin}}}} f[\alpha] \cdot g[\beta] & \text{($T_1$, involving thin $\alpha$'s)}\\
                    & \quad + \sum_{S \in \binom{[n]}{\delta n}} \textcolor{BrickRed}{\sum_{\substack{\alpha, \beta\\ \Col(\alpha \cdot \beta) = S \\ \text{$\beta$ thin}}}} f[\alpha] \cdot g[\beta] & \text{($T_2$, involving thin $\beta$'s)}\\
                    & \quad + \sum_{S \in \binom{[n]}{\delta n}} \sum_{\substack{\alpha, \beta\\ \Col(\alpha \cdot \beta) = S \\ \text{$\alpha,\beta$ wide}}} f[\alpha] \cdot g[\beta]. & \text{($T_3$, the rest)}                                                           
\end{align*}
We will address the three terms seperately and show that each of them are substantially smaller than $\Gamma(P)$.
It is easier to handle the thin terms first, as this follows just from the fact that the number of such terms is small.
\begin{claim}
  $T_1 + T_2 \leq 2^{-\Omega(n)} \cdot \Gamma(P)$. 
\end{claim}
\begin{proof}
  The main point is that there are not-too-many ways of splitting a monomial touching $\delta n$ columns into two parts where one is thin.
  Fix a set $S$ and let us estimate $T_1$ (the term $T_2$ is identical).
  The number of different ways of choosing $\alpha$ and $\beta$ that cover columns $S$ with $\alpha$ being thin is at most
  \begin{align*}
    & \;\; \binom{\delta n}{< (1 - \epsilon)\delta n} \cdot \inparen{(1- \epsilon)\delta n}^{\abs{\Row(f)}} \cdot \inparen{\delta n}^{\abs{\Row(g)}}\\
    \leq & \;\; \binom{\delta n}{< (1 - \epsilon)\delta n} \cdot \textcolor{BrickRed}{(1 - \epsilon)^{n/3}} \cdot (\delta n)^n \\
    \leq & \;\; 2^{-\Omega(n)} \cdot (\delta n)^n\\
    \leq & \;\; 2^{-\Omega(n)} \cdot F_{n,\delta n}. &\text{(\autoref{prop:bound-on-onto-fns})}
  \end{align*}
  Summing over all such $S$, and using the fact that $f[\alpha] \cdot g[\beta] < P[\alpha \cdot \beta] = 2^{-\delta n}$, we have
  \begin{align*}
    T_1 + T_2 & \leq \binom{n}{\delta n} \cdot 2^{- \Omega(n)} \cdot F_{n,\delta n} \cdot 2^{-\delta n} \\
    & \leq 2^{-\Omega(n)} \cdot \Gamma(P) & \text{(\autoref{lemma:monVPVNP-ub-for-poly})}
  \end{align*}
\end{proof}

\noindent
We are now left with the tricky part of estimating $T_3$.
The rough intuition is the following.
If we pick two ``typical'' monomials $\alpha$ and $\beta$, neither of which are thin, then $\alpha \cdot \beta$ would touch about $2(1-\epsilon)\delta n$ columns as they would likely be disjoint.
The coefficients of such monomials in $P$ is just $2^{-2(1-\epsilon)\delta n}$.
Since we must ensure that $f[\alpha] \cdot g[\beta] = P[\alpha\cdot \beta] = 2^{-2(1-\epsilon)\delta n}$ for all such $\alpha,\beta$, both $f$ and $g$ must \emph{typically} assign no more than $2^{-(1-\epsilon)\delta n}$ for $f[\alpha], g[\beta]$. But in that case, when we pick \emph{typical} $\alpha,\beta$ that touch the same set of $\delta n$ columns, then $f[\alpha] \cdot g[\beta]$ would be about $2^{-2(1-\epsilon)\delta n}$ though $P[\alpha \cdot \beta] = 2^{-(1-\epsilon)\delta n}$ which is much larger.

On the other, if $f[\alpha]$ is much large for a specific $\alpha$, then we are forced to have $g[\beta]$ quite small for any $\beta$ that touches columns disjoint from $\alpha$. Eventually we wish to ``cover'' the coefficient of monomials of $P$ that touch $\delta n$ columns as much as possible.
Hence $f$ and $g$ have conflicting requirements on what should be assigned to its coefficients, and we want to exploit this.

Without loss of generality, we may assume (by scaling $f$ and $g$ by scalars if required) that
\[
  \max_{\substack{\alpha \in f \\ \delta n \geq \abs{\Col(\alpha)} \geq (1-\epsilon)\delta n}} f[\alpha] \quad=\quad   \max_{\substack{\beta \in g \\ \delta n \geq \abs{\Col(\beta)} \geq (1-\epsilon)\delta n}}  g[\beta] \quad =: M 
\]
\paragraph{Case $1$: $M \leq 2^{-(1-\epsilon) \delta n}$.} In this case, we can plug this in our bounds directly.
\begin{align*}
  T_3 & = \sum_{S \in \binom{[n]}{\delta n}} \sum_{\substack{\alpha, \beta\\ \Col(\alpha \cdot \beta) = S \\ \text{$\alpha,\beta$ wide}}} \textcolor{BrickRed}{f[\alpha] \cdot g[\beta]} \quad \leq\quad \binom{n}{\delta n} \cdot (\delta n)^n\cdot \textcolor{BrickRed}{2^{-2(1-\epsilon)\delta n}} \\
      & \leq \binom{n}{\delta n} \cdot \frac{1}{2} \cdot F_{n,\delta n} \cdot \textcolor{BrickRed}{2^{-2(1-\epsilon)\delta n}} & \text{(\autoref{prop:bound-on-onto-fns})}\\
      & \leq \binom{n}{\delta n} \cdot 2^{-\delta n} \cdot F_{n,\delta n} \cdot 2^{-\Omega(\delta n)}\\
      & \leq 2^{-\Omega(\delta n)} \cdot \Gamma(P_1) & \text{(\autoref{lemma:monVPVNP-ub-for-poly})}
\end{align*}

\paragraph{Case $2$: $M > 2^{-(1-\epsilon) \delta n}$.} Let $\alpha_0 \in f$ and $\beta_0\in g$ be such that
\[
f[\alpha_0], g[\beta_0] \geq 2^{-(1-\epsilon) \delta n}.
\]
Note that $\Col(\alpha_0 \cdot \beta_0) \leq 2\delta n$ and hence there are many columns that are untouched by $\alpha_0$ or $\beta_0$. Therefore, if we pick a random $\delta n$ sized set $S$ at random, we should expect to have an  intersection of at most $(2\delta)\abs{S}$ with $\Col(\alpha_0 \beta_0)$. The following is a straightforward application of the Chernoff's bound.
\begin{claim}
  \[
    \Pr_{S \in \binom{[n]}{\delta n}}\insquare{\abs{S \cap \Col(\alpha_0 \beta_0)} > (2\delta + \epsilon)\abs{S} } \leq 2^{-\Omega(\delta n)}.
  \]
\end{claim}
We will call a set $S$ to be \emph{typical} if $\abs{S \cap \Col(\alpha_0 \beta_0)} \leq (2\delta + \epsilon)\delta n$, and \emph{atypical} otherwise. The above claim quantifies the fact that there aren't too many \emph{atypical} sets. Then,
\begin{align*}
  T_3 & = \sum_{S \in \binom{[n]}{\delta n}} \sum_{\substack{\alpha, \beta\\ \Col(\alpha \cdot \beta) = S \\ \text{$\alpha$,$\beta$ not thin}}} f[\alpha] \cdot g[\beta]\\
      & = \sum_{\substack{S \in \binom{[n]}{\delta n}\\\text{typical}}} \sum_{\substack{\alpha, \beta\\ \Col(\alpha \cdot \beta) = S \\ \text{$\alpha$,$\beta$ not thin}}} \textcolor{BrickRed}{f[\alpha] \cdot g[\beta]}  \quad+\quad \textcolor{BrickRed}{\sum_{\substack{S \in \binom{[n]}{\delta n}\\\text{atypical}}}} \sum_{\substack{\alpha, \beta\\ \Col(\alpha \cdot \beta) = S \\ \text{$\alpha$,$\beta$ not thin}}} f[\alpha] \cdot g[\beta]\\
      & \leq \sum_{\substack{S \in \binom{[n]}{\delta n}\\\text{typical}}} \sum_{\substack{\alpha, \beta\\ \Col(\alpha \cdot \beta) = S \\ \text{$\alpha$,$\beta$ not thin}}} \textcolor{BrickRed}{f[\alpha] \cdot g[\beta]}  \quad+\quad \textcolor{BrickRed}{\sum_{\substack{S \in \binom{[n]}{\delta n}\\\text{atypical}}}} \sum_{\substack{\alpha, \beta\\ \Col(\alpha \cdot \beta) = S \\ \text{$\alpha$,$\beta$ not thin}}} P[\alpha \cdot \beta]\\
  & \leq \sum_{\substack{S \in \binom{[n]}{\delta n}\\\text{typical}}} \sum_{\substack{\alpha, \beta\\ \Col(\alpha \cdot \beta) = S \\ \text{$\alpha$,$\beta$ not thin}}} \textcolor{BrickRed}{f[\alpha] \cdot g[\beta]}  \quad+\quad 2^{-\Omega(\delta n)} \cdot \Gamma(P)
\end{align*}
Note that, for any typical $S$, we have that $\Col(\alpha_0)$ intersects $S$ at at most $(2\delta + \epsilon)\delta n$. Therefore, $\Col(\alpha_0)$ has at least $(1 - \epsilon - (2\delta - \epsilon))\delta n$ columns that are not in $\Col(\beta)$. 
\[
  g[\beta] \leq \frac{P[\alpha_0 \beta]}{f[\alpha_0]} \leq 2^{-(1 - 2\epsilon - 2\delta)\delta n} \leq 2^{-\frac{2\delta n}{3}},
\]
and a similar bound applies for $f[\alpha]$. Therefore,
\begin{align*}
  T_3 & \leq \sum_{\substack{S \in \binom{[n]}{\delta n}\\\text{typical}}} \sum_{\substack{\alpha, \beta\\ \Col(\alpha \cdot \beta) = S \\ \text{$\alpha$,$\beta$ not thin}}} 2^{-\frac{\delta n}{3}}\cdot 2^{-\delta n} \quad+\quad 2^{-\Omega(\delta n)} \Gamma(P)\\
  & \leq 2^{-\frac{\delta n}{3}} \cdot \Gamma(P) \quad + \quad 2^{-\Omega(\delta n)}\cdot \Gamma(P) = 2^{-\Omega(\delta n)} \Gamma(P).
\end{align*}
This completes the bound for {\bf Case 2}, and thus concludes the proof of \autoref{lemma:monVPVNP-ub-for-poly} and hence \autoref{thm:monVPVNPseparation}.


%%% Local Variables:
%%% mode: latex
%%% TeX-master: "fancymain"
%%% End:
