\documentclass[11pt]{article}

\usepackage{fullpage}
\usepackage{amssymb}
\usepackage{amsmath}
\usepackage{amsthm}
\usepackage[colorlinks]{hyperref}
%\usepackage{complexity}
%\usepackage{graphicx}
%\usepackage{enumerate}

\hypersetup{
  citecolor={blue}
}


\usepackage{palatino}
\usepackage{todonotes}
%\usepackage{euler}

% %% BEGIN font stuff
% \usepackage[T1]{fontenc}
% \usepackage[urw-garamond]{mathdesign}
% \usepackage{textcomp}
% %% END font stuff 
\reversemarginpar


\newcommand{\FM}{\mathrm{FM}}
\newcommand{\Mon}{\mathrm{Mon}}


\newcommand{\rank}{\mathrm{rank}}
\newcommand{\CM}[1]{\Gamma^{\mathrm{[#1]}}}
\newcommand{\spaced}[1]{\quad#1\quad}
\newcommand{\bezout}{b\'{e}zout}
\newcommand{\Bezout}{B\'{e}zout}
\newcommand{\SPD}[3]{\inangle{\partial^{=#1}\inparen{#3}}_{\leq #2}}





%%% Circuit classes
\newcommand{\SPS}{\Sigma\Pi\Sigma}
\newcommand{\SPSP}{\Sigma\Pi\Sigma\Pi}
\newcommand{\mySPSP}[2]{\Sigma\Pi^{[#1]}\Sigma\Pi^{[#2]}}



%% \newenvironment{problem}[1]{\stepcounter{problem}
%% \paragraph{Problem \theproblem. #1}

%% }


\newtheorem{theorem}{Theorem}
\newtheorem{theorem*}{Theorem}
\newtheorem{axiom}{Axiom}
\newtheorem{corollary}[theorem]{Corollary}
\newtheorem{lemma}[theorem]{Lemma}
\def\lemmaautorefname{Lemma}
\newtheorem{observation}[theorem]{Observation}
\newtheorem{conjecture}[theorem]{Conjecture}
\newtheorem{proposition}[theorem]{Proposition}
\newtheorem{definition}[theorem]{Definition}
\newtheorem{claim}[theorem]{Claim}
\newtheorem{remark}[theorem]{Remark}
\newtheorem*{remark*}{Remark}
\newtheorem*{nclaim}{Claim}
\newtheorem{fact}[]{Fact}
\newtheorem{subclaim}[theorem]{Subclaim}
\newtheorem{problem}[]{Problem}
\newtheorem{openproblem}[]{Open Problem}

\newenvironment{myproof}[1]%
{\vspace{1ex}\noindent{\em Proof.}\hspace{0.5em}\def\myproof@name{#1}}%
{\hfill{\tiny \qed\ (\myproof@name)}\vspace{1ex}}


%Useful for recalling theorems. (hat-tip texexchange)
\makeatletter
\newtheorem*{rep@theorem}{\rep@title}
\newcommand{\newreptheorem}[2]{%
\newenvironment{rep#1}[1]{%
 \def\rep@title{#2 \ref{##1} (restated)}%
 \begin{rep@theorem}}%
 {\end{rep@theorem}}}
\makeatother

\newreptheorem{theorem}{Theorem}
\newreptheorem{lemma}{Lemma}
\newreptheorem{conjecture}{Conjecture}
\newreptheorem{problem}{Problem}

\newenvironment{proof-sketch}{\medskip\noindent{\em Sketch of Proof.}\hspace*{1em}}{\qed\bigskip}
\newenvironment{proof-attempt}{\medskip\noindent{\em Proof attempt.}\hspace*{1em}}{\bigskip}

\newenvironment{proofof}[1]{\medskip\noindent\emph{Proof of #1.}\hspace*{1em}}{\qed\bigskip}



\newcommand{\inparen }[1]{\left(#1\right)}             %\inparen{x+y}  is (x+y)
\newcommand{\inbrace }[1]{\left\{#1\right\}}           %\inbrace{x+y}  is {x+y}
\newcommand{\insquare}[1]{\left[#1\right]}             %\insquar{x+y}  is [x+y]
\newcommand{\inangle }[1]{\left\langle#1\right\rangle} %\inangle{A}    is <A>

\newcommand{\abs}[1]{\left|#1\right|}                  %\abs{x}        is |x|
\newcommand{\norm}[1]{\left\Vert#1\right\Vert}         %\norm{x}       is ||x||

\newcommand{\fspan}[1]{\F\text{-span}\inbrace{#1}}

\newcommand{\union}{\cup}
\newcommand{\Union}{\bigcup}
\newcommand{\intersection}{\cap}
\newcommand{\Intersection}{\bigcap}

\newcommand{\ceil}[1]{\lceil #1 \rceil}
\newcommand{\floor}[1]{\lfloor #1 \rfloor}


\newcommand{\eqdef}{\stackrel{\text{def}}{=}}
\newcommand{\setdef}[2]{\inbrace{{#1}\ : \ {#2}}}      % E.g: \setdef{x}{f(x) = 0}
\newcommand{\set}[1]{\inbrace{#1}}
\newcommand{\innerproduct}[2]{\left\langle{#1},{#2}\right\rangle} %\innerproduct{x}{y} is <x,y>.
\newcommand{\zo}{\inbrace{0,1}}                        % Well just something that is used often!
\newcommand{\parderiv}[2]{\frac{\partial #1}{\partial #2}}
\newcommand{\pderiv}[2]{\partial_{#2}\inparen{#1}}
\newcommand{\zof}[2]{\inbrace{0,1}^{#1}\longrightarrow \inbrace{0,1}^{#2}}


% Commonly used blackboard letters
\newcommand{\FF}{\mathbb{F}}
\newcommand{\F}{\mathbb{F}}
\newcommand{\N}{\mathbb{N}}
\newcommand{\Q}{\mathbb{Q}}
\newcommand{\Z}{\mathbb{Z}}
\newcommand{\R}{\mathbb{R}}
\newcommand{\C}{\mathbb{C}}
\newcommand{\RR}{\mathbb{R}}
\newcommand{\E}{\mathbb{E}}

\newcommand{\zigzag}{\textcircled{z}}  % for the zig-zag product


% \newcommand{\char}{\textrm{char}}
% \newcommand{\rank}{\textrm{rank}}
% \newcommand{\dim}{\textrm{dim}}



%% accented words
\newcommand{\Hastad}{H{\aa}stad }
\newcommand{\Godel}{G\"{o}del }
\newcommand{\Mobius}{M\"{o}bius }
\newcommand{\Gauss}{Gau{\ss} }
\newcommand{\naive}{na\"{\i}ve }
\newcommand{\Naive}{Na\"{\i}ve }
\newcommand{\grobner}{gr\"{o}bner }

\newcommand{\Det}{\mathsf{Det}}
\newcommand{\Perm}{\mathsf{Perm}}
\newcommand{\ESym}{\mathrm{Esym}}
\newcommand{\PSym}{\mathrm{Pow}}

%% Bold letters
\newcommand{\veca}{\mathbf{a}}
\newcommand{\vecb}{\mathbf{b}}
\newcommand{\vecc}{\mathbf{c}}
\newcommand{\vecd}{\mathbf{d}}
\newcommand{\vece}{\mathbf{e}}
\newcommand{\vecf}{\mathbf{f}}
\newcommand{\vecg}{\mathbf{g}}
\newcommand{\vech}{\mathbf{h}}
\newcommand{\veci}{\mathbf{i}}
\newcommand{\vecj}{\mathbf{j}}
\newcommand{\veck}{\mathbf{k}}
\newcommand{\vecl}{\mathbf{l}}
\newcommand{\vecm}{\mathbf{m}}
\newcommand{\vecn}{\mathbf{n}}
\newcommand{\veco}{\mathbf{o}}
\newcommand{\vecp}{\mathbf{p}}
\newcommand{\vecq}{\mathbf{q}}
\newcommand{\vecr}{\mathbf{r}}
\newcommand{\vecs}{\mathbf{s}}
\newcommand{\vect}{\mathbf{t}}
\newcommand{\vecu}{\mathbf{u}}
\newcommand{\vecv}{\mathbf{v}}
\newcommand{\vecw}{\mathbf{w}}
\newcommand{\vecx}{\mathbf{x}}
\newcommand{\vecy}{\mathbf{y}}
\newcommand{\vecz}{\mathbf{z}}



%%RP: Personal preferences 
\parindent 0pt
\renewcommand{\epsilon}{\varepsilon}

% for inline commenting by authors
\newcommand{\authnote}[2]{{\color{red}\{$<<<${\bf \footnotesize #1 notes: #2}$>>>$\}}}
\newcommand{\Rnote}[1]{\todo[color=green!40,size=\tiny]{RP: #1}}
\newcommand{\Nnote}[1]{\todo[color=blue!30,size=\tiny]{N: #1}}
\newcommand{\Rnoteinline}[1]{\todo[color=green!40,inline,size=\tiny]{RP: #1}}
\newcommand{\Nnoteinline}[1]{\todo[color=blue!30,inline,size=\tiny]{N: #1}}

\title{A selection of lower bounds for arithmetic circuits}

\author{Neeraj Kayal\\
\normalsize{Microsoft Research India}\\
\normalsize{\tt{neeraka@microsoft.com}}
\and Ramprasad Saptharishi\\
\normalsize{Microsoft Research India}\\
\normalsize{{\tt ramprasad@cmi.ac.in}}
}

\begin{document}

\maketitle

\begin{minipage}[t]{0.2\textwidth}
  \hfill
  \hfill
\end{minipage}
\begin{minipage}[t]{0.8\textwidth}
  \begin{flushright}
    \begin{quote}
      {\em It is convenient to have a measure of the 
        amount of work involved in a computing process, 
        even though it be a very crude one ... We might, 
        for instance, count the number of additions, 
        subtractions, multiplications, divisions, 
        recordings of numbers,...}
    \end{quote}
  \end{flushright}
  \begin{flushright}
    from \em{Rounding-off errors in matrix processes}, Alan M. Turing, 1948.
  \end{flushright}
\end{minipage}


\section{Introduction}\label{sec:introduction}
Polynomials originated in classical mathematical studies concerning
geometry and solutions to systems of equations. They feature in many
classical results in algebra, number theory and geometry - e.g. in
Galois and Abel's resolution of the solvability via radicals of a
quintic, Lagrange's theorem on expressing every natural number as a
sum of four squares and the impossibility of trisecting an angle
(using ruler and compass). In modern times, computer scientists began
to investigate as to what functions can be (efficiently)
computed. Polynomials being a natural class of functions, one is
naturally lead to the following question:
	\begin{center}
		\begin{minipage}{0.8\textwidth}
			{\em What is the optimum way to compute a given (family of) polynomial(s)?}
		\end{minipage}
	\end{center}
	Now the most natural way to compute a polynomial $f(x_1, x_2, \ldots, x_n) $ 
	over a field $\FF$ is to start with the input variables 
	$x_1, x_2, \ldots, x_n$ and then apply a sequence of basic 
	operations such as additions, subtractions and 
	multiplications\footnote{
		One can also allow more arithmetic operations such as division 
		and square roots. It turns out however that one can efficiently 
		simulate any circuit with divisions and square roots by another 
		circuit without these operations while incurring only an 
		polynomial factor increase in size.
	}
	in order to obtain the desired polynomial $f$. Such a computation 
	is called a straight line program. We often represent such a 
	straight-line program graphically as an arithmetic circuit - 
	wherein the overall computation corresponds to a directed acylic 
	graph whose source nodes are labelled with the input variables 
	$\{ x_1, x_2, \ldots, x_n \}$ and the internal nodes are labelled 
	with either $+$ or $\times$ (each internal node corresponds to one 
	computational step in the straight-line program). We typically allow 
	arbitrary constants from the underlying field on the incoming edges 
	of a $+$ gate so that a $+$ gate can in fact compute an arbitrary 
	$\FF$-linear combination of its inputs. The complexity of the 
	computation corresponds to the number of operations, also called 
	the size of the corresponding arithmetic circuit.
	With arithmetic circuits being the relevant model, the informal 
	question posed above can be formalized by defining the optimal 
	way to compute a given polyomial as the smallest arithmetic 
	circuit in terms of the size that computes it. While different aspects 
	of polynomials have been studied extensively in various areas of 
	mathematics, what is unique to computer science is the endeavour to 
	prove upper and lower bounds on the size of arithmetic circuits 
	computing a given (family of) polynomials. Here we give a biased 
	survey of this area, focusing mostly on lower bounds. Note that
	there are already two excellent surveys of this area - one by Avi 
	Wigderson \cite{aviSurvey} and the other by Amir Shpilka and Amir 
	Yehudayoff \cite{sy}\footnote{
		A more specialized survey by Chen, Kayal and Wigderson \cite{ckw11}
		focuses on the applications of partial derivatives in understanding 
		the structure and complexity of polynomials. 
	}. Our intention in writing the survey is the underlying hope 
	that revisiting and assimilating the known results pertaining to 
	circuit lower bounds will in turn help us make progress on this 
	beautiful problem. Consequently we mostly present here those 
	results which we for some reason felt we did not understand 
	comprehensively enough. We conclude with some recent lower 
	bound results for homogeneous bounded depth formulas. Some 
	notable lower bound results that we are unable to present 
	here due to space and time constraints are as follows. A 
	quadratic lower bound for depth three circuits by Shpilka and Wigderson \cite{sw2001}, for bounded occur 
	bounded depth formulas by Agrawal, Saha, Saptharishi and 
	Saxena \cite{ASSS12} and the $n^{1 + \Omega(1/r)}$ lower bound for 
	circuits of depth $r$ by Raz \cite{raz10}.\\

\noindent {\bf Overview.}
The state of affairs in arithmetic complexity is such that despite a
lot of attention we still have only modest lower bounds for general
circuits and formulas. In order to make progress, recent work has
focused on restricted subclasses.  We first present the best known
lower bound for general circuits due to Baur and Strassen \cite{BS83},
and a lower bound for formulas due to Kalorkoti
\cite{k85}. The subsequent lower bounds that we present
follow a common roadmap and we articulate this in Section
\ref{sec:roadmap}, and present some simple lower bounds to help the
reader gain familiarity. We then present (a slight generalization of)
an exponential lower bound for monotone circuits due to Jerrum and
Snir \cite{js82}.  Moving on to more restricted (but still nontrivial
and interesting) models, we first present an exponential lower bound
for depth three circuits over finite fields due to Grigoriev and
Karpinski \cite{grigoriev98} and multilinear formulas. We conclude
with some recent progress on lower bounds for homogeneous depth four
circuits.
	 
\begin{remark*} Throughout the article, we shall use $\Det_n$ and $\Perm_n$ to refer to the determinant and permanent respectively of a symbolic $n\times n$ matrix $\inparen{\!\inparen{x_{ij}}\!}_{1\leq i,j\leq n}$. 

\end{remark*}

	
	
	
	

%%% Local Variables: 
%%% mode: latex
%%% TeX-master: "lowerbounds"
%%% End: 


\section{Existential lower bounds}\label{sec:random}
Before we embark on our quest to prove lower bounds for interesting
families of polynomials, it is natural to ask as to what bounds
one can hope to achieve.  For a multivariate polynomial 
$f(\vecx) \in \FF[\vecx]$, denote by $S(f)$ the size of 
the smallest arithmetic circuit computing
$f$.  
\begin{theorem}
	{\bf [Folklore.]}
  For ``most'' polynomials $f(\vecx) \in \FF[\vecx]$ of degree $d$ 
  on $n$ variables we have 
  $$ S(f) \spaced{\geq} \Omega \inparen{\sqrt{\binom{n + d}{d}}}. $$
\end{theorem}
\begin{proof-sketch}
  We prove this here only in the situation where the underlying field
  $\FF$ is a finite field and refer the reader to another survey
  (\cite{ckw11}, Chapter 4) for a proof in the general case. So let
  $\FF = \FF_{q}$ be a finite field. Any line of a straight line
  program computing $f$ can be expressed as taking the product of two
  $\F_q$-linear combinations of previously computed values. Hence the
  total number of straight-line programs of length $s$ is at most
  $q^{O(s^2)}$. On the other hand there are $q^{\binom{n+d}{d}}$
  polynomials of degree $d$ on $n$ variables.  Hence most $n$-variate
  polynomials of degree $d$ require straight-line programs of length
  at least (equivalently arithmetic circuits of size at least) $s =
  \Omega \inparen{\sqrt{\binom{n+d}{d}}}$.
	\end{proof-sketch}
	
\noindent	Hrubes and Yehudayoff \cite {hy11} showed that in fact 
	most $n$-variate polynomials of degree $d$ {\em with zero-one 
	coefficients} have complexity at least $\Omega\inparen{\sqrt{\binom{n+d}{d}}}$. 
	Now it turns out that this is in fact a lower bound on the 
	number of multiplications in any circuit computing a random 
	polynomial. Lovett \cite{lovett11} complements this 
	nicely by giving a matching upper bound. Specifically, 
	it was shown in \cite{lovett11} that for any polynomial $f$ 
	of degree $d$ on $n$ variables there exists a circuit computing 
	$f$ having at most $\inparen{\sqrt{\binom{n + d}{d}}} \cdot (nd)^{O(1)} $ 
	multiplications.



%%% Local Variables: 
%%% mode: latex
%%% TeX-master: "lowerbounds"
%%% End: 


\section{Weak lower bounds for general circuits and formulas}\label{sec:gen-ckt-formulas}

Despite several attempts by various researchers to prove lower bounds for arithmetic circuits or formulas, we only have very mild lower bounds for general circuits or formulas thus far. In this section, we shall look at the two  modest lower bounds for general circuits and formulas. 

\subsection{Lower bounds for general circuits}\label{sec:baur-strassen}

The only super-linear lower bound we currently know for general arithmetic circuits is the following  result of Baur and Strassen \cite{BS83}.

\begin{theorem}[\cite{BS83}]\label{thm:baur-strassen}
  Any fan-in $2$ circuit that computes the polynomial $f = x_1^{d+1} + \dots + x_n^{d+1}$ has size $\Omega(n\log d)$. 
\end{theorem}

\subsubsection{An exploitable weakness}

Each gate of the circuit $\Phi$ computes a local operation on the two children. To formalize this, define a new variable $y_g$ for every gate $g \in \Phi$. Further, for every gate $g$ define a quadratic equation $Q_g$ as
$$
Q_g = \begin{cases} y_g - (y_{g_1} + y_{g_2}) & \text{if $g = g_1 + g_2$}\\
  y_g - (y_{g_1}\cdot y_{g_2}) & \text{if $g = g_1 \cdot g_2$}.
\end{cases}
$$
Further if $y_o$  corresponds to the output gate, then the system of equations
$$\setdef{Q_g = 0}{g\in \Phi} \spaced{\union} \inbrace{y_{o} = 1}$$
completely characterize the computations of $\Phi$ that results in an output of $1$. 

The same can also be extended for \emph{multi-output} circuits that compute several polynomials simultaneously. In such cases, the set of equations
$$\setdef{Q_g = 0}{g\in \Phi} \spaced{\union} \setdef{y_{o_i} = 1}{i=1, \ldots, n}$$
completely characterize computations that result in an output of all ones. The following classical theorem allows us to bound the number of  common roots to a system of polynomial equations. 

\begin{theorem}[\Bezout's theorem]
  Let $g_1,\dots, g_r \in \F[X]$ such that $\deg(g_i) = d_i$ such that the number of common roots of $g_1=\dots=g_r = 0$ is finite. Then, the number of common roots (counted with multiplicities) is bounded by $\prod d_i$.
\end{theorem}

Thus in particular, if we have a circuit $\Phi$ of size $s$ that \emph{simultaneously} computes $\inbrace{x_1^d, \dots,x_n^d}$, then we have $d^n$ inputs that evaluate to all ones (where each $x_i$ must be  a $d$-th root of unity). Hence, \Bezout's theorem implies that
$$
2^s\spaced{\geq} d^n \spaced{\quad\implies\quad} s \spaced{=} \Omega(d\log n).
$$

Observe that $\inbrace{x_1^d,\dots, x_n^d}$ are all first-order derivatives of $f = x_1^{d+1}+\dots+x_n^{d+1}$ (with suitable scaling). A natural question here is the following --- if $f$ can be computed an arithmetic circuit of size $s$, what is the size required to compute all first-order partial derivatives of $f$ simultaneously? The \naive approach of computing each derivative separately results in a circuit of size $O(s\cdot n)$. Baur and Strassen \cite{BS83} show that we can save a factor of $n$.

\begin{lemma}[\cite{BS83}]\label{lem:baur-strassen}
  Let $\Phi$ be an arithmetic circuit of size $s$ and fan-in $2$ that computes a polynomial $f\in \F[X]$. Then, there is a multi-output circuit  of size $O(s)$ computing all first order derivatives of $f$.
\end{lemma}

Note that this immediately implies that any circuit computing $f = x_1^{d+1} + \dots + x_n^{d+1}$ requires size $\Omega(d\log n)$ as claimed by Theorem~\ref{thm:baur-strassen}. 


\subsubsection{Computing all first order derivatives simultaneously}

Since we are working with fan-in $2$ circuits, the number of edges is at most twice the size. Hence let $s$ denote the number of edges in the circuit $\Phi$, and we shall prove by induction that all first order derivatives of $\Phi$ can be computed by a circuit of size at most $5s$. Pick a non-leaf node $v$ in the circuit $\Phi$ closest to the leaves with both its children being variables, and say $x_1$ and $x_2$ are the variables feeding into $v$. In other words, $v = x_1 \odot x_2$ where $\odot$ is either $+$ or $\times$.

Let $\Phi'$ be the circuit obtained by deleting the two edges feeding into $v$, and replacing $v$ by a new variable. Hence, $\Phi'$ computes a polynomial $f' \in \F[X\union \inbrace{v}]$ and has at most $(s-1)$ edges. By induction on the size, we can assume that there is a circuit $\mathbb{D}(\Phi')$ consisting of at most $5(s-1)$ edges that computes all the first order derivatives of $f'$.

Observe that since $f'\mid_{(v = x_1 \odot x_2)} = f(\vecx)$,  we have that 
$$
\parderiv{f}{x_i} \spaced{=}\inparen{\parderiv{f'}{x_i}}_{v = x_1 \odot x_2} \quad+\quad  \inparen{\parderiv{f'}{v}}_{v = x_1 \odot x_2}\inparen{\parderiv{(x_1 \odot x_2)}{x_i}}.
$$

Hence, if $v = x_1 + x_2$ then
\begin{eqnarray*}
  \parderiv{f}{x_1} & = & \inparen{\parderiv{f'}{x_1}}_{v=x_1 + x_2} +\quad \inparen{\parderiv{f'}{v}}_{v = x_1 + x_2}\\
  \parderiv{f}{x_2} & = & \inparen{\parderiv{f'}{x_2}}_{v=x_1 + x_2} +\quad \inparen{\parderiv{f'}{v}}_{v = x_1 + x_2}\\
  \parderiv{f}{x_i} & = & \inparen{\parderiv{f'}{x_i}}_{v=x_1 + x_2} \qquad\text{for $i>2$}.
\end{eqnarray*}
If $v = x_1 \cdot x_2$, then
\begin{eqnarray*}
  \parderiv{f}{x_1} & = & \inparen{\parderiv{f'}{x_1}}_{v=x_1 \cdot x_2} + \inparen{\parderiv{f'}{v}}_{v = x_1 \cdot x_2} \cdot x_2\\
  \parderiv{f}{x_2} & = & \inparen{\parderiv{f'}{x_2}}_{v=x_1 \cdot x_2} + \inparen{\parderiv{f'}{v}}_{v = x_1\cdot x_2}\cdot x_1\\
  \parderiv{f}{x_i} & = & \inparen{\parderiv{f'}{x_i}}_{v=x_1 \cdot x_2} \qquad\text{for $i>2$}.
\end{eqnarray*}

Hence, by adding at most $5$ additional edges to $\mathbb{D}(\Phi')$, we can construct $\mathbb{D}(\Phi)$ and hence size of $\mathbb{D}(\Phi)$ is at most $5s$. \qed (Lemma~\ref{lem:baur-strassen})

\subsection{Lower bounds for formulas}\label{sec:Kalorkoti}

This section would be devoted to the proof of Kalorkoti's lower bound
\cite{k85} for formulas computing $\Det_n$, $\Perm_n$.

\begin{theorem}[\cite{k85}]\label{thm:kalorkoti}
  Any arithmetic formula computing $\Perm_n$ (or $\Det_n$) requires
  $\Omega(n^3)$ size.
\end{theorem}

The exploitable weakness in this setting is again to use the fact that the polynomials computed at intermediate gates share many polynomial dependencies. 

\begin{definition}[Algebraic independence]
  A set of polynomials $\inbrace{f_1,\dots, f_m}$ is said to be
  \emph{algebraically independent} if there is no non-trivial polynomial 
  $H(z_1,\dots, z_m)$ such that $H(f_1,\dots, f_m)=0$. 

  The size of the largest algebraically independent subset of
  $\vecf=\inbrace{f_1,\dots, f_m}$ is called the \emph{transcendence
    degree} (denoted by $\mathrm{trdeg}(f)$).
\end{definition}

The proof of Kalorkoti's theorem proceeds by defining a \emph{complexity measure} using the above notion of algebraic independence. \\


{\bf The Measure:} 
For any subset of variables $Y\subseteq X$, we can write a polynomial
$f \in \F[X]$ of the form $f = \sum_{i=1}^s f_i \cdot m_i$ where $m_i$'s are
distinct monomials in the variables in $Y$, and $f_i \in
F[X \setminus Y]$. We shall denote by $\mathrm{td}_Y(f)$ the transcendence degree of $\inbrace{f_1,\dots, f_s}$


Fix a partition of variables $X = X_1 \sqcup \dots
\sqcup X_r$. For any polynomial $f\in \F[X]$, define the map $\CM{Kal}:\F[X]\rightarrow \Z_{\geq 0}$  as
$$
\CM{Kal}(f) \quad=\quad \sum_{i=1}^r \mathrm{td}_{X_i}(f).
$$

The lower bound proceeds in two natural steps:
\begin{enumerate}
\item Show that $\CM{Kal}(f)$ is \emph{small} whenever $f$ is computable by a \emph{small} formula. 
\item Show that $\CM{Kal}(\Det_n)$ is \emph{large}. 
\end{enumerate}

\subsubsection{Upper bounding $\CM{Kal}$ for a formula}

\begin{lemma}\label{lem:kal-upperbound}
  Let $f$ be computed by a fan-in two formula $\Phi$ of size $s$. Then
  for any partition of variables $X = X_1\sqcup \dots \sqcup X_r$, we
  have $\CM{Kal}(f) = O(s)$.
\end{lemma}
\begin{proof}
  For any node $v\in \Phi$, let $\textsc{Leaf}(v)$ denote the leaves
  of the subtree rooted at $v$ and let $\textsc{Leaf}_{X_i}(v)$ denote
  the leaves of the subtree rooted at $v$ that are in the part
  $X_i$. Since the underlying graph of $\Phi$ is a tree, it follows
  that the size of $\Phi$ is bounded by a twice the number of
  leaves. For each part $X_i$, we shall show that
  $\mathrm{td}_{X_i}(f) = O(\abs{\textsc{Leaf}_{X_i}(\Phi)})$, which
  would prove the required bound. \\

  Fix an arbitrary part $Y = X_i$. Define the following three 
  sets of nodes
  \begin{eqnarray*}
    V_0 & = & \setdef{v\in \Phi}{\abs{\textsc{Leaf}_{Y}(v)} = 0 \quad\text{and}\quad \abs{\textsc{Leaf}_{Y}(\textsc{Parent}(v))} \geq 2}\\
    V_1 & = & \setdef{v\in \Phi}{\abs{\textsc{Leaf}_{Y}(v)} = 1 \quad\text{and}\quad \abs{\textsc{Leaf}_{Y}(\textsc{Parent}(v))} \geq 2}\\
    V_2 & = & \setdef{v\in \Phi}{\abs{\textsc{Leaf}_{Y}(v)} \geq 2}.
  \end{eqnarray*}

  Each node $v\in V_0$ computes a polynomial in $f_v \in
  \F[X\setminus Y]$, and we shall replace the subtree at $v$ by a node
  computing the polynomial $f_v$. Similarly, any node $v\in V_1$
  computes a polynomial of the form $f^{(0)}_v + f^{(1)}_v y_v$ for some $y_v\in Y$
  and $f^{(0)}_v, f^{(1)}_v \in \F[X\setminus Y]$. We shall again replace the
  subtree rooted at $v$ by a node computing $f^{(0)}_v + f^{(1)}_v y_v$. 

  Hence, the formula $\Phi$ now reduces to a smaller formula $\Phi_Y$ with
  leaves  being the nodes in $V_0$ and $V_1$ (and nodes in $V_2$ are
  unaffected). We would like to show that the size of the reduced
  formula, which is at most twice the number of its leaves, is
  $O(\abs{\textsc{Leaf}_Y(\Phi)})$.

  \begin{observation}\label{obs:v1-bound}$\abs{V_1} \leq \abs{\textsc{Leaf}_Y(\Phi)}$.    
  \end{observation}
  \begin{myproof}{Obs}
    Each node in $V_1$ has a distinct leaf labelled with a variable in
    $Y$. Hence, $\abs{V_1}$ is bounded by the number of leaves
    labelled with a variable in $Y$.
  \end{myproof}
  
  This shows that the $V_1$ leaves are not too many. Unfortunately, we
  cannot immediately bound the number of $V_0$ leaves, since we could
  have a long chain of $V_2$ nodes each with one sibling being a $V_0$
  leaf. The following observation would show how we can eliminate such
  long chains.

  \begin{observation}\label{obs:same-leaf-collapse}
    Let $u$ be an arbitrary node, and $v$ be another node in the
    subtree rooted at $u$ with $\textsc{Leaf}_Y(u) =
    \textsc{Leaf}_Y(v)$. Then the polynomial $g_u$ computed at $u$ and
    the polynomial $g_v$ computed at $v$ are related as $g_u = f_1 g_v
    + f_2$ for some $f_1,f_2 \in \F[X\setminus Y]$.
  \end{observation}
  \begin{myproof}{Obs}
    If $\textsc{Leaf}_Y(u) =\textsc{Leaf}_Y(v)$, then every node on
    the path from $u$ to $v$ must have a $V_0$ leaf as the other child. The
    observation follows as all these nodes are $+$ or $\times$ gates.
  \end{myproof}

  Using the above observation, we shall remove the need for $V_0$
  nodes completely by augmenting each node $u$ (computing a polynomial
  $g_u$) in $\Phi_Y$ with polynomials $f_u^{(0)}, f_u^{(1)} \in \F[X\setminus Y]$
  to enable them to compute $f_u^{(1)}g_u + f_u^{(0)}$. Let this augmented formula be called $\hat{\Phi}_Y$. Using
  Observation~\ref{obs:same-leaf-collapse}, we can now contract any
  two nodes $u$ and $v$ with $\textsc{Leaf}_Y(u) =
  \textsc{Leaf}_Y(v)$, and eliminate all $V_0$ nodes
  completely. Since all $V_2$ nodes are internal nodes, the only leaves of the augmented formula $\hat{\Phi}_Y$ are in $V_1$. Hence, the size of the augmented formula $\hat{\Phi}_Y$ is  bounded by $2\abs{V_1}$, which is bounded by
  $2\abs{\textsc{Leaf}_Y(\Phi)}$ by Observation~\ref{obs:v1-bound}.\\

  Suppose $\Phi$ computes a polynomial $f$, which can be written as  $f
  = \sum_{i=1}^t f_i\cdot m_i$ with $f_i \in \F[X\setminus Y]$ and $m_i$'s being
  distinct monomials in $Y$. Since $\hat{\Phi}_Y$ also computes $f$, each $f_i$ is a polynomial
  combination of the polynomials $S_Y = \setdef{f_{u}^{(0)},
    f_{u}^{(1)}}{u\in \hat{\Phi}_Y}$. Since $\hat{\Phi}_Y$ consists of at
  most $2\abs{\textsc{Leaf}_Y(\Phi)}$ augmented nodes, we have that
  $\mathrm{td}_Y(f) \leq |S_Y| \leq 4\abs{\textsc{Leaf}_Y(\Phi)}$. Therefore, 
  $$
  \mathrm{td}_Y(f) \quad=\quad \mathrm{trdeg}\setdef{f_i}{i\in [t]} \quad\leq\quad 4\abs{\textsc{Leaf}_Y(\Phi)}
  $$
  Hence, 
  $$\CM{Kal}(\Phi) = \sum_{i=1}^r \mathrm{td}_{X_i}(f_i) \leq 4\inparen{\sum_{i=1}^r \abs{\textsc{Leaf}_{X_i}}} = O(s).
  $$
\end{proof}

\subsubsection{Lower bounding $\CM{Kal}(\Det_n)$}


\begin{lemma}\label{lem:kal-lowerbound}
  Let $X = X_1 \sqcup \dots \sqcup X_n$ be the partition as defined by
  $X_t = \setdef{x_{ij}}{i-j\equiv t\bmod{n}}$. Then,
  $\CM{Kal}(\Det_n) = \Omega(n^3)$.
\end{lemma}
\begin{proof}
  By symmetry, it is easy to see that $\text{td}_{X_i}(\Det_n)$ is
  the same for all $i$. Hence, it suffices to show that
  $\text{td}_{Y}(\Det_n) = \Omega(n^2)$ for $Y = X_n = \inbrace{x_{11},\dots, x_{nn}}$. 

  To see this, observe that the determinant consists of the monomials
  $\inparen{\frac{x_{11}\dots x_{nn}}{x_{ii}x_{jj}}}\cdot
  x_{ij}x_{ji}$ for every $i\neq j$. Hence, $\text{td}_{Y}(\Det_n)
  \geq \mathrm{trdeg}\setdef{x_{ij}x_{ji}}{i\neq j} =
  \Omega(n^2)$. Therefore, $\CM{Kal}(\Det_n) =
  \Omega(n^3)$.
\end{proof}

The proof of Theorem~\ref{thm:kalorkoti} follows from
Lemma~\ref{lem:kal-upperbound} and Lemma~\ref{lem:kal-lowerbound}.


%%% Local Variables: 
%%% mode: latex
%%% TeX-master: "lowerbounds_birk"
%%% End: 



%%% LOCAL Variables: 
%%% mode: latex
%%% TeX-master: "lowerbounds"
%%% End: 


\section{``Natural'' proof strategies}\label{sec:roadmap}

The lower bounds presented in Section~\ref{sec:gen-ckt-formulas} proceeded by first identifying a \emph{weakness} of the model, and exploiting it in an explicit manner. More concretely, Section~\ref{sec:Kalorkoti} presents a promising strategy that could be adopted to prove lower bounds for various models of arithmetic circuits. The crux of the lower bound was the construction of a good map $\Gamma$ that assigned a number to every polynomial. The map $\CM{Kal}$ was useful to show a lower bound in the sense that any $f$ computable by a \emph{small} formula had \emph{small} $\CM{Kal}(f)$. In fact, all subsequent lower bounds in arithmetic circuit complexity have more or less followed a similar template of a ``natural proof''. More concretely, all the subsequent lower bounds we shall see would essentially follow the outlined plan.  

\begin{quote}
{\bf Step 1 (normal forms)} For every circuit in the circuit class $\mathcal{C}$ of interest, express the polynomial computed as a \emph{small sum of simple building blocks}. 
\end{quote}

For example, every $\Sigma\Pi\Sigma$ circuit is a \emph{small} sum of \emph{products of linear polynomials} which are the building blocks here. In this case, the circuit model naturally admits such a representation but we shall see other examples with very different representations as sum of building blocks. 

\begin{quote}
{\bf Step 2 (complexity measure)} Construct a map $\Gamma: \F[x_1,\dots, x_n] \rightarrow \Z_{\geq 0}$ that is \emph{sub-additive} i.e. $\Gamma(f_1 + f_2)\leq \Gamma(f_1) + \Gamma(f_2)$.
\end{quote}

In most cases, $\Gamma(f)$ is the rank of a large matrix whose entries are linear functions in the coefficients of $f$. In such cases, we immediately get that $\Gamma$ is sub-additive. 

The strength of the choice of $\Gamma$ is determined by the next step. 

\begin{quote}
{\bf Step 3 (potential usefulness)} Show that if $B$ is a \emph{simple building block}, then $\Gamma(B)$ is \emph{small}.
Further, check if $\Gamma(f)$ for a \emph{random polynomial} $f$ is large (potentially). 
\end{quote}

This would suggest that if any $f$ with large $\Gamma(f)$ is to be written as a sum of $B_1 + \dots + B_s$, then sub-additivity and the fact that $\Gamma(B_i)$ is small for each $i$ and $\Gamma(f)$ is large immediately imply that $s$ must be large. This implies that the complexity measure $\Gamma$ does indeed have a potential to prove a lower bound for the class. The next step is just to replace the \emph{random polynomial} by an explicit polynomial. 

\begin{quote}
{\bf Step 4 (explicit lower bound)} Find an explicit polynomial $f$ for which $\Gamma(f)$ is large. 
\end{quote} 



These are usually the steps taken in almost all the known arithmetic circuit lower bound proofs. The main ingenuity lies in constructing a useful complexity measure, which is really to design $\Gamma$ so that it is small on the \emph{building blocks}. \\

Of course, there could potentially be lower bound proofs that do not follow the road-map outlined. For instance, it could be possible that $\Gamma$ is not small for a random polynomial, but specifically tailored in a way to make $\Gamma$ large for the $\Perm_n$. Or perhaps $\Gamma$ need not even be sub-additive and maybe there is a very different way to argue that all polynomial in the circuit class have small $\Gamma$. However, this has been the road-map for almost all lower bounds so far (barring very few exceptions). As a warmup, we first present some very simple applications of the above plan to prove lower bounds for some very simple subclasses of arithmetic circuits in the next section. We then move on to more sophisticated proofs of lower bounds for less restricted subclasses of circuits. 


%%% Local Variables: 
%%% mode: latex
%%% TeX-master: "lowerbounds"
%%% End: 


\section{Some simple lower bounds}

Let us start with the simplest complete\footnote{in the sense that any polynomial can be computed in this model albeit of large size}  class of arithmetic circuits -- depth-$2$ circuits or $\Sigma\Pi$ circuits. 

\subsection{Lower bounds for $\Sigma\Pi$ circuits}

Any $\Sigma\Pi$ circuit of size $s$ computes a polynomial $f = m_1 + \dots + m_s$ where each $m_i$ is a monomial multiplied by a field constant. Therefore, any polynomial computed by a \emph{small} $\Sigma\Pi$ circuit must have a \emph{small} number of monomials. Hence, it is obvious that any polynomial that has many monomials require large $\Sigma\Pi$ circuits. 

This can be readily rephrased in the language of the outline described last section by defining $\Gamma(f)$ to simply be the number of monomials present in $f$. Hence, $\Gamma(f)\leq s$ for any $f$ computed by a $\Sigma\Pi$ circuit of size $s$. Of course, even a polynomial like $f = (x_1 + x_2+\dots + x_n)^n$  have $\Gamma(f) = n^{\Omega(n)}$ giving the lower bound. 

\subsection{Lower bounds for $\Sigma\!\wedge\!\Sigma$ circuits}

A $\Sigma\!\wedge\!\Sigma$ circuit of size $s$ computes a polynomial of the form $f = \ell_1^{d_1} + \dots + \ell_s^{d_s}$ where each $\ell_i$ is a linear polynomial over the $n$ variables.\footnote{such circuits are also called \emph{diagonal depth-$3$ circuits} in the literature}

Clearly as even a single $\ell^d$ could have exponentially many monomials, the $\Gamma$ defined above cannot work in this setting. Nevertheless, we shall try to design a similar map to ensure that $\Gamma(f)$ is \emph{small} whenever $f$ is computable by a \emph{small} $\Sigma\!\wedge\!\Sigma$ circuit. \\

In this setting, the \emph{building blocks} are terms of the form $\ell^d$. The goal would be to construct a \emph{sub-additive} measure $\Gamma$ such that $\Gamma(\ell^d)$ is \emph{small}. Here is the key observation to guide us towards a good choice of $\Gamma$. 

\begin{observation}
Any $k$-th order partial derivative of $\ell^d$ is a constant multiple of $\ell^{d-k}$. 
\end{observation}

Hence, if $\partial^{=k}(f)$ denotes the set of $k$-th order partial derivatives of $f$, then the space spanned by $\partial^{=k}(\ell^d)$ has dimension $1$. This naturally leads us to define $\Gamma$ exploiting this weakness. 

$$
\Gamma_k(f)\quad \eqdef \quad \dim \inparen{\partial^{=k}(f)}
$$

It is straightforward to check that $\Gamma_k$ is indeed sub-additive and hence $\Gamma_k(f) \leq s$ whenever $f$ is computable by a $\Sigma\!\wedge\!\Sigma$ circuit of size $s$. For a random polynomial $f$, we should be expecting $\Gamma_k(f)$ to be $\binom{n+k}{k}$ as there is unlikely to be any linear dependencies among the partial derivatives. Hence, all that needs to be done is to find an explicit polynomial with large $\Gamma_k$. 


If we consider $\Det_n$ or $\Perm_n$, then any partial derivative of order $k$ is just an $(n-k)\times(n-k)$ minor. Also, these minors consist of disjoint sets of monomials and hence are linearly independent. Hence, $\Gamma_k(\Det_n) = \binom{n}{k}^2$. Choosing $k = n/2$, we immediately get that any $\Sigma\!\wedge\!\Sigma$ circuit computing $\Det_n$ or $\Perm_n$ must be of size $2^{\Omega(n)}$. \\

\subsection{Low-rank $\Sigma\Pi\Sigma$}\label{sec:low-rank-sps}

A slight generalization of $\Sigma\!\wedge\!\Sigma$ circuits is a \emph{rank-$r$ $\Sigma\Pi\Sigma$ circuit} that computes a polynomial of the form 
$$
f \spaced{=}  T_1 \;+\; \dots \;+\; T_s
$$
where each $T_i = \ell_{i1}\dots \ell_{id}$ is a product of linear polynomials such that $\dim\inbrace{\ell_{i1},\dots, \ell_{id}}\leq r$. \\

Thus, $\Sigma\!\wedge\!\Sigma$  is a rank-$1$ $\Sigma\Pi\Sigma$ circuit, and a similar partial-derivative technique for lower bounds works here as well. 

In the setting where $r$ is much smaller than the number of variables $n$, each $T_i$ is essentially an $r$-variate polynomial masquerading as an $n$-variate polynomial using an affine transformation. In particular, the set of $n$ first order derivatives of $T$ have rank at most $r$. This yields the following observation.

\begin{observation}
Let $T = \ell_1\dots \ell_d$ with $\dim\inbrace{\ell_1,\dots, \ell_d}\leq r$. Then for any $k$, we have
$$
\Gamma_k(T) \spaced{\eqdef}\dim\inparen{\partial^{=k}(T)} \spaced{\leq} \binom{r+k}{k}.
$$
\end{observation}

Thus once again by sub-additivity, for any polynomial $f$ computable by a rank-$r$ $\Sigma\Pi\Sigma$ circuit of size $s$, we have $\Gamma_k(f) \leq s\cdot \binom{r+k}{r}$. Note that a random polynomial is expected to have $\Gamma_k(f)$ close to $\binom{n+k}{k}$, which could be much larger for $r\ll n$. We already saw that $\Gamma_k(\Det_n) = \binom{n}{k}^2$. This immediately gives the following lower bound, the proof of which we leave as an exercise to the interested reader. 

\begin{theorem}\label{thm:low-rank-sps-lb}
Let $r \leq n^{2-\delta}$ for some constant $\delta > 0$. For $k = \epsilon n^{\delta}$, where $\epsilon > 0$ is sufficiently small, we have
$$
\frac{\binom{n}{k}^2}{\binom{r+k}{k}} \quad=\quad \exp\inparen{\Omega(n^{\delta})}.
$$
Hence, any rank-$r$ $\Sigma\Pi\Sigma$ circuit computing $\Det_n$ or $\Perm_n$ must have size $\exp\inparen{\Omega(n^\delta)}$. \qed
\end{theorem}


This technique of using the rank of partial derivatives was introduced by Nisan and Wigderson \cite{nw1997} to prove lower bounds for \emph{homogeneous depth-$3$ circuits} (which also follows as a corollary of Theorem~\ref{thm:low-rank-sps-lb}). The survey of Chen, Kayal and Wigderson \cite{ckw11} give a comprehensive exposition of the power of the \emph{partial derivative method}. \\



With these simple examples, we can move on to other lower bounds for various other more interesting models. 



%%% Local Variables: 
%%% mode: latex
%%% TeX-master: "lowerbounds_birk"
%%% End: 


\section{Lower bounds for monotone circuits}

This section would present a slight generalization 
	of a lower bound by Jerrum and Snir~\cite{js82}. 
	To motivate our presentation here, let us first 
	assume that the underlying field is $\RR$, the 
	field of real numbers. A monotone circuit over 
	$\RR$ is a circuit having $+, \times$ gates in 
	which all the field constants are {\em non-negative} 
	real numbers. Such a circuit can compute any 
	polynomial $f$ over $\RR$ all of whose 
	coefficients are nonnegative real numbers, such as 
	for example the permanent. It is then natural to ask 
	whether there are small monotone circuits over $\RR$ 
	computing the permanent. Jerrum and Snir \cite{js82}
	obtained an exponential lower bound on the size 
	of monotone circuits over $\RR$ computing the 
	permanent. Note that this	definition of monotone 
	circuits is valid only over $\RR$ (actually more 
	generally over ordered fields but not over say finite 
	fields) and such circuits can only compute polynomials 
	with non-negative coefficients. Here we will present 
	Jerrum and Snir's argument in a slightly more 
	generalized form such that the circuit model makes 
	sense over any field $\FF$ and is complete, i.e. 
	can compute any polynomial over $\FF$. Let us first 
	explain the motivation behind the generalized circuit
	model that we present here. Observe that in any monotone 
	circuit over $\RR$, there is no cancellation as there 
	are no negative coefficients. Formally, for a node $v$ 
	in our circuits let us denote by $f_{v}$ the polynomial 
	computed at that node. For a polynomial $f$ let us denote 
	by $\Mon(f)$ the set of monomials having a nonzero 
	coefficient in the polynomial $f$. 
		\begin{enumerate}
		
			\item If $w = u + v$ then 
				$$ \Mon(f_{w}) = \Mon(f_{u}) \cup \Mon(f_{v}). $$
			
			\item If $w = u \times v$ then
				$$ \Mon(f_{w}) = \Mon(f_{u}) \cdot \Mon(f_{v}) \eqdef \setdef{m_1\cdot m_2}{m_1 \in \Mon(f_{u}), m_2\in \Mon(f_{v})}. $$
			
		\end{enumerate}
	This means that for any node $v$ in a monote circuit over 
	$\RR$ one can determine $\Mon(f_{v})$ in a very syntactic 
	manner starting from the leaf nodes. Let us make precise 
	this syntactic computation that we have in mind.
	
	\begin{definition}[Formal Monomials.]
  Let $\Phi$ be an arithmetic circuit. The \emph{formal monomials} at
  any node $v\in \Phi$, which shall be denoted by $\FM(v)$, shall 
  be inductively defined as follows:
	  \begin{quote}
	    If $v$ is a leaf labelled by a variable $x_i$, then $\FM(v) =
	    \inbrace{x_i}$. If it is labelled by a constant, then $\FM(v) =
	    \inbrace{1}$.
	    
	    If $v = v_1 + v_2$, then $\FM(v) = \FM(v_1) \union \FM(v_2)$. 
	
	    If $v = v_1 \times v_2$, then 
            \begin{eqnarray*}
              \FM(v) &=& \FM(v_1)\cdot \FM(v_2)\\
              &\eqdef& \setdef{m_1\cdot m_2}{m_1 \in \FM(v_1), m_2\in \FM(v_2)}.
            \end{eqnarray*}
          \end{quote}
	\end{definition}
	
\noindent Note that for any node $v$ in any circuit 
	we have $\Mon(f_{v}) \subseteq \FM(v)$ but in a 
	monotone circuit over $\RR$ this containment is in 
	fact an equality at every node. This motivates our 
	definition of a slightly more general notion of a 
	monotone circuit as follows. 

	
\begin{definition}[Monotone circuits]
  A circuit $C$ is said to be
  \emph{syntactically monotone}
   (simply monotone for short) if $\Mon(f_{v}) = \FM(v)$ 
   for every node $v$ in $C$.
\end{definition}
	
	
	
The main theorem of this section is the following: 

\begin{theorem}[\cite{js82}]\label{thm:monotone-circuit-lbs}
	Over any field $\FF$, any syntactically monotone circuit 
	$C$ computing $\Det_n$ or $\Perm_n$ must have
  size at least $2^{\Omega(n)}$.
\end{theorem}

The proof of this theorem is relatively short assuming the
	following structural result (which is present in standard
	depth-reduction proofs \cite{vsbr83,ajmv98}).

\begin{lemma}\label{lem:vsbr-two-thirds}
  Let $f$ be a degree $d$ polynomial computed by a monotone circuit of
  size $s$. Then, $f$ can be written of the form $f = \sum_{i=1}^s f_i
  \cdot g_i$ where the $f_i$'s and $g_i$'s satisfy the following
  properties.
\begin{enumerate}
\item For each $i\in [s]$, we have $\frac{d}{3} < \deg{g_i} \leq
  \frac{2d}{3}$.
\item For each $i$, we have $\FM(f_i)\cdot \FM(g_i) \subseteq \FM(f)$.
\end{enumerate}
\end{lemma}

We shall defer this lemma to the end of the section and first see how
this would imply Theorem~\ref{thm:monotone-circuit-lbs}. The complexity measure $\Gamma(f)$ in this case is just the number of monomials in $f$, but it is the above \emph{normal form} that is crucial in the lower bound. 

\begin{proofof}{Theorem~\ref{thm:monotone-circuit-lbs}}
Suppose $\Phi$ is a circuit of size $s$ that computes $\Det_n$. Then by Lemma~\ref{lem:vsbr-two-thirds}, 
$$
\Det_n \quad=\quad \sum_{i=1}^s f_i \cdot g_i
$$
with $\FM(f_i)\cdot \FM(g_i) \subseteq \FM(\Det_n)$. The building blocks are terms of the form $T = f\cdot g$, where $\FM(f)  \cdot \FM(g) \subseteq \FM(\Det_n)$. \\

Since all the monomials in $\Det_n$ are products of variables from
distinct columns and rows, the rows (and columns) containing the
variables $f$ depends on is disjoint from the rows (and columns)
containing variables that $g$ depends on. Hence, there exists sets of indices $A,B
\subseteq [n]$ such that $f$ depends only on $\setdef{x_{jk}}{j\in
  A, k\in B}$ and $g$ depends only on
$\setdef{x_{jk}}{j\in\overline{A}, k\in
  \overline{B}}$.

Further, since $\Det_n$ is a homogeneous polynomial of degree $n$, we also have that both $f$ and $g$ must be homogeneous as well. Also as all monomials of $g$ using distinct row and column indices from $\overline{A}$ and $\overline{B}$ respectively, we see that $\deg g = |\overline{A}| = |\overline{B}|$ and $\deg f = |A| = |B|$. Using Lemma~\ref{lem:vsbr-two-thirds}, let $|A| = \alpha n$ for some $\frac{1}{3}\leq \alpha \leq \frac{2}{3}$. This implies that $\Gamma(f)\leq (\alpha n)!$, and $\Gamma(g)\leq ((1-\alpha)n)!$ and hence
$$\Gamma(f\cdot g) \quad \leq\quad (\alpha n)! ((1-\alpha)n)! \spaced{\leq} \frac{n!}{\binom{n}{n/3}} $$
as $\frac{1}{3}\leq \alpha \leq \frac{2}{3}$.
Also, $\Gamma$ is clearly sub-additive and we have $$\Gamma(f_1g_1 + \dots + f_s g_s) \spaced{\leq} s \cdot \frac{n!}{\binom{n}{n/3}}.$$
Since $\Gamma(\Det_n) = n!$, this forces $s \geq
\binom{n}{n/3} = 2^{\Omega(n)}$.
\end{proofof}

We only need to prove Lemma~\ref{lem:vsbr-two-thirds} now.

\subsection{Proof of Lemma~\ref{lem:vsbr-two-thirds}}

Without loss of generality, assume that the circuit $\Phi$ is
homogeneous\footnote{It is a forklore result that any circuit can be \emph{homogenized} with just a polynomial blow-up in size. Further, this process also preserves monotonicity of the circuit. A proof of this may be seen in \cite{sy}.}, and consists of alternating layers of $+$ and $\times$
gates. Also, assume that all $\times$ gates have fan-in two, and
orient the two children such that the formal degree of the left child
is at least as large as the formal degree of the right child. Such
circuits are also called \emph{left-heavy} circuits.

\begin{definition}[Proof tree]\label{defn:prooftree}
A \emph{proof tree} of an arithmetic circuit $\Phi$ is a sub-circuit $\Phi'$ such that
\begin{itemize}
\item The root of $\Phi$ is in $\Phi'$
\item If a multiplication gate with $v = v_1\times v_2 \in \Phi'$, then $v_1$ and $v_2$ are in $\Phi'$ as well.
\item If an addition gate $v = v_1 + \dots + v_s \in \Phi'$, then
  exactly one $v_i$ is in $\Phi'$.
\end{itemize}
Such a sub-circuit $\Phi'$, represented as a tree (duplicating nodes if
required), shall be called a \emph{proof tree} of $\Phi$.
\end{definition}

\newcommand{\PF}{\textsc{ProofTrees}}

Let $\PF(\Phi)$ denote the set of all proof trees of $\Phi$. It is
easy to see that any proof tree of $\Phi$ computes a monomial over the
variables. Further, if $\Phi$ was a monotone circuit computing a polynomial
$f$, then every proof tree computes a monomial in $f$. Therefore,
$$
f\quad=\quad \sum_{\Phi' \in \PF(\Phi)} [\Phi']
$$
where $[\Phi']$ denotes the monomial computed by $\Phi'$. Of course, the number of proof trees is exponential and the above
expression is huge. However, we could use a divide-and-conquer
approach to the above equation using the following lemma. 

\begin{lemma}\label{lem:brent-two-thirds}
  Let $\Phi'$ be a left-heavy formula of formal degree $d$. Then there
  is a node $v$ on the left-most path of $\Phi'$ such that
  $\frac{d}{3}\leq \deg(v)< \frac{2d}{3}$.
\end{lemma}
\begin{proof}
  Pick the lowest node on the left-most path that has degree at least
  $\frac{2d}{3}$. Then, its left child must be a node of degree less
  than $\frac{2d}{3}$, and also at least $\frac{d}{3}$ (because the
  formula is left-heavy). 
\end{proof}

For any proof tree $\Phi'$ and a node $v$ on its left-most path,
define $[\Phi':v]$ to be the output polynomial of the proof tree 
obtained by replacing the node $v$ on the left-most path by $1$. 
If $v$ does not occur on the
left-most path of $\Phi'$, define $[\Phi':v]$ to be $0$. We will 
denote the polynomial computed at a node $v$ by $f_{v}$. Then, the
above equation can now be re-written as:
\begin{eqnarray*}
f & = & \sum_{\Phi' \in \PF(\Phi)} [\Phi']\\
  & = & \sum_{\substack{v\in \Phi\\ \frac{d}{3}\leq \deg{v} < \frac{2d}{3}}} f_{v} \cdot \inparen{ \sum_{\Phi'\in \PF(\Phi)} [\Phi':v]}\\
  & = & \sum_{\substack{v\in \Phi\\ \frac{d}{3}\leq \deg{v} < \frac{2d}{3}}} f_{v} \cdot g_v\quad\quad\text{where $g_v = \sum_{\Phi'\in \PF(\Phi)} [\Phi':v]$} .\\
\end{eqnarray*}
Since $\frac{d}{3} \leq \deg{v} < \frac{2d}{3}$, we also have that
$\frac{d}{3} < \deg{g_v} \leq \frac{2d}{3}$ and the last equation is
what was required by Lemma~\ref{lem:vsbr-two-thirds}.\qed


%%% Local Variables: 
%%% mode: latex
%%% TeX-master: "lowerbounds_birk"
%%% End: 


\input{secGK}

\input{secMultilinear}

\input{secShiftedPartials}

\section{Conclusion}\label{sec:conclusion}
	The field of arithmetic complexity, like Boolean complexity, 
	abounds with open problems and proving lower bounds for 
	almost any natural subclass of arithmetic circuits is 
	interesting especially if the currently known techniques/
	complexity measures do not apply to that subclass\footnote{
		Some of the complexity measures that we describe here 
		yield lower bounds for slightly more general subclasses 
		of circuits.
	}. 
	The surveys \cite{aviSurvey, sy, ckw11} mark out the 
	frontiers of this area in the form of many open problems 
	and we invite the reader to try some of them.
	
	

%%% Local Variables: 
%%% mode: latex
%%% TeX-master: "lowerbounds_birk"
%%% End: 


\bibliographystyle{alpha}
\bibliography{references}

\end{document}
