\section{Lower bounds for depth-3 circuits}

\subsection{Depth-3 circuits over finite fields}

This section shall present the lower bound of Grigoriev and Karpinski
\cite{grigoriev98} for $\Det_n$. A follow-up paper of Grigoriev and
Razborov \cite{gr00} extended the result over function fields, also
including a weaker lower bound for the permanent, but we shall just
present the lower bound for the determinant.

\begin{theorem}\cite{grigoriev98}\label{thm:gk-main-thm}
  Any depth-3 circuit computing $\Det_n$ over a finite field $\F_q$
  requires size $2^{\Omega(n)}$.
\end{theorem}

{\bf Main idea:} Let $q = |\F|$. Suppose $C = T_1 + \dots + T_s$,
where each $T_i$ is a product of linear polynomials. Define
$\mathrm{rank}(T_i)$ to be the dimension of the set of linear
polynomials that $T_i$ is a product of. If $T$ is a product of linear
forms of rank at least $r$, then with high probability a random
evaluation of $T$ would be zero. Hence, in a sense, $C$ can be
``approximated'' by just the low-rank components. Further, if $T =
\ell_1^{e_1} \dots \ell_d^{e_d}$ with $\ell_1,\dots, \ell_r$ being a
maximal set of linearly independent linear polynomials, then $T$
interpretted as a function from $\F$ to $\F$ can be written as a
linear combination of $\setdef{\ell_1^{d_1}\dots\ell_r^{d_r}}{d_i < q
  \quad \forall i\in [d]}$. Any linear combination of the above set is
equivalent to a polynomial with at most $q^r$ monomials upto a linear
transformation; we shall refer to them as \emph{generalized
  $q^r$-sparse polynomials}. Hence, any depth-3 circuit $C$ over $\F$
can be \emph{approximated} by a sum of generalized sparse polynomials. 

Grigoriev and Karpinski \cite{grigoriev98} formalize the above idea as
a measure, and then proceed to show that $\Det_n$ cannot be
approximated by such sum of generalized sparse polynomials.

\subsubsection{The complexity measure}

For any polynomial $f \in \F[x_{11},\dots, x_{nn}]$, define the matrix
$M_k(f)$ as follows --- the columns of $M_k(f)$ are indexed by $k$-th
order partial derivatives of $f$, and rows by elements of $\F^{n^2}$,
with the entry being the evaluation of the partial derivative (column
index) at the point (row index).\\

The rank of $M_k(f)$ could be a possible choice of a complexity
measure but Grigoriev and Karpinski make a small
modification. Instead, the look at the matrix $M_k(f)$ and remove a
few \emph{erraneous} evaluation points and use the rank of the
resulting matrix. For any $\mathcal{A}\subseteq \F^{n^2}$, let us
define $M_k(f;\mathcal{A})$ to be the matrix obtained from $M_k(f)$ by
only taking the rows whose indices are in
$\mathcal{A}$. Also, let $\CM{GK}_{k,\mathcal{A}}(f)$
denote $\mathrm{rank}(M_k(f;\mathcal{A}))$.



\subsubsection{Upper-bounding $\CM{GK}_{k,\mathcal{A}}$ for a depth-3 circuit}\label{sec:gk-upper-bound}

The following observation is trivial to observe from the linearity of the entries of the matrix. 

\begin{observation}[Sub-additivity]\label{obs:GK-subadditivity}
  $\CM{GK}_{k,\mathcal{A}}(f + g) \quad\leq\quad
  \CM{GK}_{k,\mathcal{A}}(f) +
  \CM{GK}_{k,\mathcal{A}}(g)$
\end{observation}

Fix a threshold $r_0 = \beta n$ for some constant $\beta > 0$ (to be
chosen shortly), and let $k = \gamma n$ for some $\gamma>0$ (to be
chosen shortly). We shall call a term $T = \ell_1\cdots \ell_d$ to be
of \emph{low rank} if $\mathrm{rank}(T) \leq r$, and \emph{large rank}
otherwise. By the above lemma, we shall upper-bound the
$\CM{GK}_{k,\mathcal{A}}$ for each term $T$, for a
suitable choice of $\mathcal{A}$.\\

\noindent 
{\bf Low rank terms:}

Suppose $T = \ell_1 \cdots \ell_d$ with $\inbrace{\ell_1, \dots,
  \ell_r}$ being a maximal independent set of linear polynomials (with
$r \leq r_0$). Then, $T$ can be expressed as a linear combination of
terms from the set $\setdef{\ell_1^{e_1}\dots \ell_r^{e_r}}{e_i\leq d
  \quad \forall i\in [r]}$. And since the matrix $M_k(f)$ depends only
on evaluations in $\F^{n^2}$, we can use the relation that $x^q = x$
in $\F$ to express the function $T:\F^{n^2}\rightarrow \F$ as a linear
combination of $\setdef{\ell_1^{e_1}\dots \ell_r^{e_r}}{e_i<q \quad
  \forall i\in [r]}$. Therefore, for any set $\mathcal{A}\subseteq
\F^{n^2}$, we have that
$$
\CM{GK}_{k;\mathcal{A}}(T) \quad \leq \quad
\mathrm{rank}(M_k(f)) \quad \leq \quad q^r \quad\leq \quad q^{\beta{n}}
$$


\noindent
{\bf High rank terms:}

Suppose $T = \ell_1\dots \ell_d$ whose rank is greater than $r_0 =
\beta n$, and let $\inbrace{\ell_1, \dots, \ell_r}$ be a maximal
independent set. We want to use the fact that since $T$ is a product
of at least $r$ independent linear polynomials, most evaluations would
be zero. We shall be choosing our $\mathcal{A}$ to be the set where
all $k$-th order partial derivatives evaluate to zero. 

On applying the product rule of differentiation, any $k$-th order of
$T$ can be written as a sum of terms each of which is a product of at
least $r-k$ independent linear polynomials. Let us count the
\emph{erraneous points} $\mathcal{E}_T\subseteq \F^{n^2}$ that keep at
least $r-k$ of $\inbrace{\ell_1,\dots, \ell_r}$ non-zero, or in other
words makes at most $k$ of $\inbrace{\ell_1,\dots, \ell_r}$ zero.
$$
\Pr_{\mathbf{a}\in \F^{n^2}}[\text{at most $k$ of $\ell_1,\dots,
  \ell_r$ evaluate to zero}] \quad\leq \quad \sum_{i=0}^k \binom{r}{i} \inparen{\frac{1}{q}}^i\inparen{1 - \frac{1}{q}}^{r-i}
$$
Hence, we can upper-bound $\abs{\mathcal{E}_T}$ as
\begin{eqnarray*}
\abs{\mathcal{E}_T} &\leq & \sum_{i=0}^k \binom{r}{i} (q-1)^{r-i}q^{n^2 - r}\\
 & = & O\inparen{k\cdot \binom{r}{k}\inparen{1 - \frac{1}{q}}^{r-k} q^{n^2}} \quad\text{if $r > qk$}\\
 & = & q^{n^2} \cdot \alpha^n\quad\text{for some $0< \alpha < 1$}
 \end{eqnarray*}

 By choosing $\mathcal{A} = \F_{n^2} \setminus \mathcal{E}$ where
 $\mathcal{E} = \Union_{\text{$T$ of large rank}} \mathcal{E}_t$, we
 have that $M_k(T;\mathcal{A})$ is just the zero matrix and hence
 $\CM{GK}_{k,\mathcal{A}}(T) = 0$.\\


Putting it together, if $C = T_1 + \dots + T_s$, then 
\begin{equation}\label{eqn:gk-upper-bound-circuit}
\CM{GK}_{k,\mathcal{A}}(C) \quad \leq \quad s \cdot
q^{\beta n}
\end{equation} 
where $\mathcal{A} = \F^{n^2} \setminus \mathcal{E}$ for some set
$\mathcal{E}$ of size at most $s \cdot \alpha^n \cdot q^{n^2}$ for
some $1< \alpha < 1$.

\subsubsection{Lower-bounding $\CM{GK}_{k,\mathcal{A}}(\Det_n)$}

We now wish to show that $M_k(\Det_n;\mathcal{A})$ has large
rank. Note that if we were to just consider $M_k(\Det_n)$, it would
have been easy to show that the rank is full by looking at just those
evaluation points that keep exactly $(n-k)\times (n-k)$ minor non-zero
(set the main diagonal of the minor to ones, and every other entry to
zero). Hence, $M_k(\Det_n)$ has the identity matrix \emph{embedded
  inside} and hence must be full rank. In fact, we can also show that
$M_k(\Det_n; GL_n)$ is full rank (where $GL_n$ is the set of
invertible matrices) by the following lemma.

\begin{lemma}
  If a multivariate polynomial vanishes on all invertible matrices,
  then the polynomial must be zero.
\end{lemma}

The proof is left as a nice exercise to the reader\footnote{Hint:
  induct on $n$}. Hence, if the matrix restricted to just invertible
evaluations is not full rank, then there is some linear combination of
the columns that is zero. Since any linear combination of columns is a
multilinear polynomial, this linear combination must vanish on every
invertible matrix, and hence must be zero by the above lemma. But we
know that the minors of a symbolic matrix are linearly independent
yielding the contradiction. Hence, $\mathrm{rank}(M_k(f;GL_n)) =
\binom{n}{k}^2$. But we don't have much control over the choice of
$\mathcal{E}$. Grigoriev and Karpinski circumvent this difficulty
using the following two clever tricks.

\begin{lemma}\label{lem:gk-invertible-same-rank}
  For any invertible matrix $g\in GL_n$ and any set $\mathcal{A}\subseteq \F^{n^2}$, 
  $$
  \mathrm{rank}(M_k(\Det_n;\mathcal{A})) = \mathrm{rank}(M_k(\Det_n;g\mathcal{A}))
  $$
  where $g\mathcal{A} \quad=\quad \setdef{g \cdot a}{a\in \mathcal{A}}$.
\end{lemma}
\begin{proof}
  For any matrix $g$, the minors of $gX$ are linear combinations of
  the minors of $X$ where $X = \inparen{\!\inparen{x_{ij}}\!}$ is
  symbolic matrix of indeterminates. Hence, any column of
  $M_k(\Det_n;g\mathcal{A})$ is a linear combination of columns of
  $M_k(\Det_n;\mathcal{A})$, and hence
  $\mathrm{rank}(M_k(\Det_n;\mathcal{A})) \geq
  \mathrm{rank}(M_k(\Det_n;g\mathcal{A}))$. If $g$ is an invertible
  matrix, the inequality holds the other way as well, and we have the
  lemma.
\end{proof}

Suppose we can find a small set of invertible matrices $S \subseteq
\F^{n^2}$ such that $GL_n \subseteq \Union_{g\in S} g\mathcal{A}$,
then the matrix $M_k(\Det_n; GL_n)$ is a sub-matrix of $M_k(\Det_n;
\Union_{g\in S} g\mathcal{A})$. Hence, 
\begin{eqnarray*}
  \binom{n}{k}^2 = \mathrm{rank}(M_k(\Det_n;GL_n)) & \leq & \mathrm{rank}\inparen{M_k\inparen{\Det_n;\Union_{g\in S}g\mathcal{A}}}\\
  & \leq & |S| \cdot \mathrm{rank}(M_k(\Det_n;\mathcal{A}))\quad\text{by Lemma~\ref{lem:gk-invertible-same-rank}}
\end{eqnarray*}

Hence, we now need to find a small set $S$ of invertible matrices such
that $\mathcal{A}$ can be \emph{shifted} by $S$ to cover all of
$GL_n$. Note that $\mathcal{A} = \F^{n^2} \setminus \mathcal{E}$ for
some set $\mathcal{E}$ of size $s\alpha^nq^{n^2}$ (for $\alpha <
1$). Since it is a well-known fact that $\abs{GL_n} \geq
q^{n^2}\frac{(q-2)}{(q-1)}$, we have that $\mathcal{E}' = GL_n
\intersection \mathcal{E}$ has size at most $\frac{1}{2} \abs{GL_n}$
and hence $\mathcal{A}' = \mathcal{A}\intersection GL_n$ has size at
least $\frac{1}{2}\abs{GL_n}$. We shall show that a random set $S$ of
$O(n^2)$ invertible matrices would cover all of $GL_n$. Observe that
for any matrix $E$,
\begin{eqnarray*}
  \Pr_g[E \notin g\mathcal{A}'] &\leq& \frac{1}{2}\\
  \implies \Pr_{g_1,\dots, g_m}[E \notin g_i\mathcal{A}'\text{ for every } i\in [m]] & \leq & \frac{1}{2^m}\\
  \implies \Pr_{g_1,\dots, g_m}[\exists E\in \mathcal{E}'\text{ such that }E \notin g_i\mathcal{A}'\text{ for every } i\in [m]] & \leq & \frac{\abs{\mathcal{E}'}}{2^m}
\end{eqnarray*}
By choosing $m > \log_2 \abs{\mathcal{E}'} = O(n^2)$, there exists a
set $S$ of $m$ invertible matrices that $S\mathcal{A}$ covers $GL_n$. Hence, 
$$
\mathrm{rank}(M_k(\Det_n;\mathcal{A})) \quad = \quad \Omega\inparen{\frac{\binom{n}{k}^2}{n^2}}
$$

\subsubsection{Putting it all together}

Hence, if $\Det_n$ is computed by a depth-3 circuit of top fan-in $s$ over $\F$, then 
\begin{eqnarray*}
s \cdot q^{\beta n} & = &  \Omega\inparen{\frac{\binom{n}{k}^2}{n^2}}\\
 & = & \Omega\inparen{2^{2H(\gamma)\cdot n - O(\log n)}}\\
\implies \log s & = & \Omega((2H(\gamma) - \beta \log q)n)
\end{eqnarray*}

By choosing $\gamma < q^{-q/2}$, we can find a $\beta$ such that
$q\gamma < \beta$ (which was required in
section~\ref{sec:gk-upper-bound}) and $2H(\gamma) > \beta \log q$,
yielding the required lower bound that $s = 2^{\Omega(n)}$.\qed
(Theorem~\ref{thm:gk-main-thm})


%%% Local Variables: 
%%% mode: latex
%%% TeX-master: "lowerbounds"
%%% End: 
